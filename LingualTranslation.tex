\section{Lingual translations}

Due to the nature of the Rubin Observatory, it is inevitable that some documents will need to be translated into other languages.
The major need will be English and Spanish translations for the American and Chilean operational locations.
It is a significant yet inevitable undertaking for bilingual translations throughout the observatory lifetime.
English/Spanish bilingual documents will be needed, and they must be available with up-to-date information to prevent issues during construction and maintenance.
The accuracy of language is crucial (e.g., context, terminology), from work instructions to inclusivity (e.g., engagement in continuous improvement).
The risk of improper or unavailable translations can generally impact the schedule (e.g., delay in work) and results in increased risk to personnel, equipment, the environment or security when procedures or protocols cannot be accurately followed.

The Documentation Working Group recommends that, at a minimum, the following translated in both, English and Spanish:
\begin{itemize}
	\item procedure documents and manuals for activities conducted on a routine basis,
	\item any and all safety documents, and
	\item websites that facilitate the organization and location of translated documentation (e.g. DocuShare).
\end{itemize}
Any other documents, drawings, schematics, parts lists, etc. should be translated on an as-needed basis.

Workflows for documentation should include steps to determine if translation is needed and its implement, as defined by the project of technical group.
The determination and reason for translation should be recorded; i.e., required, preferred, suggested or unneeded.
When releasing documents, at least one approver should be designated specifically to confirm that the translated version matches the original.
This is especially important when revisions are made, to ensure any changes are fully and accurately captured.
It should be clear within the document or applicable sections which have translation and the respective location(s).

When producing translations, the Documentation Working Group recommends that they are reviewed by a technically-competent bilingual person.
For version-controlled documents, it is recommended that original and translated versions be kept in a single file.
The location for translated documents can be done in whatever format is most appropriate for the document; e.g., a schematic might have both English and Spanish labels next to each other, or a procedure may be written entirely in English and then repeated in Spanish.

The construction project should be involved as much as practical, but the effort will continue into operations.
Additionally, the Documentation Working Group believes it is a requirement for the project to allocate resources for ongoing support throughout operations.
While estimating the resources needed for this, it is recommended to consider additional translation needs for global associations, including user-facing scientific documentation.
