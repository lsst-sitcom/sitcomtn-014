\subsection{Topic View}

The \emph{Topic View} aimes to categorize information into domain specific document types.
Its purpose is to allow users the ability to find information of a specific form within a documentation system (\it{e.g.} 3D CAD models).
It first requires a defined decomposition of systems, subsystems and components.
% The sentance above seems to be at odds with the narrative in the "Product View" sectiom
Topics could range to any categorization, such as physical location or department ownership.
The Topic View must be agnostic to the storage location because it requires an access portal to utilize it, such as the Rubin Documentation Portal (Section \ref{sec:DocPortal}).
The construction of a Topic View tree depends on the implementation of the portal, the purpose of the topic tree, and the targeted user base.

The Documentation Working Group suggests the following categories to define a system for the Topic View:

\begin{itemize}
\item design documents
\item requirement documents
\item technical manuals (e.g., operation or maintenance)
\item operational documents (e.g., procedures)
\item data performance
\item evaluation processes
\item maintenance reports
\item safety and hazard mitigations
\item access and software
\end{itemize}
