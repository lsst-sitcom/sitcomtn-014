\subsection{Path Forward with Solidworks Product Data Management (PDM)}
\label{sec:pdm-path}

As part of the proposal, the Documentation Working Group considered and evaluated options for Solidworks Product Data Management (PDM) Professional \citep{PDM-cite} use in operations.
It is expected it will continued to be used in operations as the official CAD model repository, as it is currently is in construction for the Telescope \& Site subsystem.
The utility provided by the configuration management and modeling software not only provides the needed historical reference, including the baseline and as-built designs, but the system and construction project files will be needed for design changes and upgrades throughout the lifecycle of the Rubin Observatory.

Since the observatory design and construction has yet to be completed, an overall CAD model of the observatory is not available at this time.
The Documentation Working Group recommends that once commissioning is complete, an overall CAD model of the observatory is produced.
As the internal systems of the observatory are complex, this initial iteration should only represent high-level systems and interactions at the beginning of operations to permit operations staff time to understand how the systems interact in a mechanical phase space.
Although, a fully detailed CAD model should be created as soon as practicable have a reliable, version-controlled reference throughout the design lifecycles taking place throughout operations.
This effort will require a full-time resource with a high-level of expertise in CAD modeling with the Solidworks software for $6$ months .

The PDM system uses a data card feature that allows searchable metadata to be associated with each file in the vault.
Vendor files not created using Solidworks do not contain the metadata used to populate these data cards.
Since converted CAD models from vendors have blank data cards, it is recommended that the final as-built CAD models imported to Solidworks file format have the metadata entered to allow easy search access.
This will require a full-time resource for an estimated time of $3$-$6$ months.

A workflow process is currently in place for all files in the PDM system.
This workflow allowed the baseline design models and ICDs to be checked by the design engineering group and approved by the systems engineering group.
Once approved for release, files are locked to avoid changes.
A workflow revision process is in place to allow for changes after the release.
It's recommended that the both workflow processes be reviewed by the operations teams so they may be modified as needed to work with systems that the operations team will use throughout the life of the observatory.

The PDM system and associated design software must be administered.
The project should continue hosting servers in both locations, Tuscon and La Serena.
The project IT group currently administers the servers and performs software installation and upgrades when needed on users' computers.
It's recommended to have a CAD and PDM vault administrator for managing the system, implementing periodic changes and assisting user's with upgrades.
This position will be needed on a part-time basis over the life of the observatory.
Management must consider if the role should be filled by internal staff or the software value-added reseller (VAR) offers contract services.

The PDM system uses a software system that uses a set of licenses to allow users to access the PDM vault.
The construction team currently uses $12$ CAD editor licenses (allows changes access to files) and $5$ viewer licenses.
Each license has a yearly maintenance fee that allows for annual upgrades as well as access to service packs for the software during the current year and technical support for the software.
It is recommended that the operations team determine how many licenses will be needed during operations to accurately estimate the annual budget and allow access with editor or viewer use to staff that require it.
