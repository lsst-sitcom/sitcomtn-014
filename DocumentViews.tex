\section{Documentation Views}
\label{sec:views}

There are four \emph{Documentation Views} with a set of customized categories and subcategories developed by the Documentation Working Group to functionally describe and organize the technical documentation for the Rubin Observatory.
The views and categories will assist those responsible for technical documentation when considering the informational needs and interests of users, communities, operations staff, and other stakeholders.
Each view is represented by a \emph{tree structure}, a widely used way of representing hierarchical structures typically in a graphical form.
Examples in this proposal include classical node-link diagrams and tree view outlines that are built with \emph{branches} of connected nodes, or \emph{leaf-nodes}, starting from \emph{root-nodes} representing the highest level in the hierarchy.
Different tree structure diagrams can be used depending on the particular use case and how to effectively represent the information.
\citep{wiki-tree-diagram-cite}

Each View has its own purpose and they are designed to provide a framework structure to search, reference and retrieve currently available information consistently project-wide.
In addition, the Views will provide an orderly way to introduce new documentation and meet an extended goal of facilitate referencing and minimize replication.

The four views are:

\begin{itemize}

  \item \textbf{Product View} --- For organizing the ownership of documentation, describing internal systems and providing structure for linking and cross referencing documentation or informational dependencies.
  Notably includes the separation of Rubin Observatory Departments.

  \item \textbf{Storage View} --- For describing all project documentation storage locations and repositories.
  In conjunction with the Product View, identifying or determining the normative source of information.

  \item \textbf{Access View} --- For describing the user base and documentation applicable to their use cases.

  \item \textbf{Topic View} --- For searching and discovering documentation.

\end{itemize}

The owner or a designated responsible group will use the Documentation Views to develop an effective way to communicate the various pieces of documentation, how they are interconnected, dependencies or connections to another group's documentation, and utilization for various documents.
Stakeholders (e.g., internal and external technical groups, Rubin observatory communities) should be able to retrieve current information with consistency and reliability.
It is the responsibility of the owner or designated responsible group to keep this information up-to-date, organized, and readily available for stakeholders.
Stakeholders should provide regularly feedback over the course of operations.

This framework will improve the ability of pre-operations and operations staff to clearly establish and review information for construction completeness and operational readiness.
For example, all identified normative sources of information should be completed; or, a use case where multiple documents referencing one normative source more effectively serves multiple stakeholders than all related information in a single document.
The framework also supports managers and auditors in understanding a relevant system.
For example, a document that is referenced isn't identified as a normative source of information; a normative source of information is not located in an appropriate repository; or, identifying interfaces between systems to ensure operational requirements are met.

The following subsections propose how each Documentation View tree is constructed.
