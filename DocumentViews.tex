\section{Documentation Views}
\label{sec:views}

There are four \emph{Documentation Views} developed with customized categories and subcategories to functionally describe and organize the technical documentation for Rubin Observatory.
The set of categories were developed by the Documentation Working Group to assist the group responsible for the documents in considering users' or communities' interest and their informational needs during observatory operations.
Each view is represented by a \emph{tree structure}, a widely used way of representing hierarchical structures typically in a graphical form.
Examples in this proposal include classical node-link diagrams and tree view outlines that are built with \emph{branches} of connected nodes, or \emph{leaf-nodes}, starting from \emph{root-nodes} representing the highest level in the hierarchy.
Different tree structure diagrams can be used depending on the particular use case and how to effectively represent the information.
\citep{wiki-tree-diagram-cite}

Each View has its own purpose.
They are designed to provide a framework structure for staff and users to search, reference and retrieve currently available information while considering their expected interests consistently project-wide.
In addition, the Views will provide an orderly way to introduce new documentation and meet an extended goal of facilitate referencing and minimize replication.

The four views are:

\begin{itemize}

  \item \textbf{Product View} --- For organizing the ownership of documentation, describing internal systems and providing structure for linking and cross referencing documentation or informational dependencies.
  Notably includes the separation of Rubin Observatory Departments.

  \item \textbf{Storage View} --- For describing all project documentation storage locations and repositories.
  In conjunction with the Product View, identifying or determining the normative source of information.

  \item \textbf{Access View} --- For describing the user base and documentation applicable to their use cases.

  \item \textbf{Topic View} --- For searching and discovering documentation.

\end{itemize}

The owner or a designated responsible group will use the Documentation Views to develop an effective way to communicate the various pieces of documentation, how they are interconnected, dependencies or connections to another group's documentation, and utilization for various documents.
Stakeholders (e.g., internal and external technical groups, Rubin observatory communities) should be able to retrieve current information with consistency and reliability.
It is the responsibility of the owner or designated responsible group to keep this information up-to-date and useful to all stakeholders.
Stakeholders should provide regularly feedback over the course of operations.

This framework will improve the ability of pre-operations and operations staff to clearly and robustly establish the criteria and requirements for construction completeness and operations readiness; for example, all identified normative sources of information should be completed, or a use case where multiple documents referencing one normative source better serves multiple stakeholders than one document.
The framework should also contribute to manager's and auditor's ability to understand the system at-hand for their consideration; for example, a document that is referenced isn't identified as a normative source of information or requires a different repository (e.g., DocuShare), or interfaces between systems should be further defined to ensure operational requirements are met.

The following subsections propose how each Documentation View tree is constructed, various examples of use, recommendations for technical teams to consider, and capture any assessments of needed resources.
