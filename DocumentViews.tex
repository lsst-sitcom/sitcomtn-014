\section{Documentation Views}
\label{sec:views}

The section describe three different view of the documentation content:

We have developed 3 views to help organized the future (e.g. delivered) state of the Rubin Construction Project documentation.  These are:

\begin{itemize}

\item The Product View;
\item The Access View; and
\item The Storage View.

\end{itemize}
 
Within each view, particularly the Access view, we have subcategorized the view into access for Users/operator, Engineers/Technical and Scientists.  The Product and Access view derive their content from the Storage view.  The Storage view is a means to determine where the source of "truth" lies.

The ''view'' are represented by ''trees'' provide structure for different areas of the project involved in the Rubin Observatory Documentation Process, which involves the Information Retrieval, distribution, and archiving to the future operation phase of the Vera C Rubin Observatory  Observatory, to provide a robust and orderly structure to future users,  to accomplish their expectations on searching and reference of documentation and at the same time, provide a structure to include the new documentation, to register in detail, the information that is being produced during the Operational Phase of the telescope Subsystems and corresponding facilities.

The different trees shall embrace the different steps that are involved in a General and wide Documentation Process Requirement, in which the different Managing Areas of the project will provide and develop their corresponding and common Operation Documentation with an approach on the users' expectation to a reliable searching process, through all the Rubin Observatory associated groups.

It is intended that the implementation of these views will constitute a deliverable from the construction project to the operations team.

It is also important to note that no content current in storage structure (see the Documentation Inventory assessment in SITCOMTN-012) will be lost.  This content will be maintained as is, but not necessary fully incorporated into the construct of the final delivered technical documentation packed as described below.

