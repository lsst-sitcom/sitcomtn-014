\section{Documentation Views}
\label{sec:views}

The Documentation Working Group developed four "Documentation~Views" with customized subcategories to functionally describe and organize the documentation for Rubin Observatory Operations.
Each view is represented by a \emph{tree~structure}, a widely used way of representing hierarchical structures in a graphical form.
This proposal includes classical node-link diagrams and tree view outlines that are built with branches of connected nodes, or "leaf nodes,"
starting from "root nodes," or starting nodes that are the highest level in the hierarchy;
however, it may be possible to use different tree structure diagrams depending on the particular use case.
\citep{wiki-tree-diagram-cite}

The four views are:

\begin{itemize}

\item The Product View --- For organizing the ownership of documentation, describing internal systems and providing structure for linking and cross referencing documentation or informational dependencies.
\item The Access View --- For describing the user base and documentation applicable to their use cases.
\item The Storage View --- For determining the normative source of information (e.g., source of truth) and listing documentation locations and/or repositories.
\item The Topic View --- For searching and discovering documentation.

\end{itemize}
 
These views are designed to provide a consistent structure for staff and users to search, reference and retrieve currently available information while considering their expected interests,
and provide a robust and orderly way to include new documentation with the goal of facilitating references to minimize replication.
The following subsections propose how each view's tree is constructed and various examples of use.

It is intended that the implementation of these views will constitute a deliverable from the Construction Project to the Operations Team.
The different managing groups will complete these views/trees to have documentation organized in common way for operations.
It is also important to note that no content currently used in construction or pre-operations will be lost
(see the Rubin Observatory Construction Documentation Inventory, \url{https://sitcomtn-012.lsst.io/}).
This content will be maintained as is for availability,
but not necessary incorporated into the documentation scheme described in this proposal.
