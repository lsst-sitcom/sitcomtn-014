\documentclass[SE,authoryear,toc]{lsstdoc}
% lsstdoc documentation: https://lsst-texmf.lsst.io/lsstdoc.html
\input{meta}

% Package imports go here.

% Local commands go here.

%If you want glossaries
%\input{aglossary.tex}
%\makeglossaries

\title{Project Documentation Future State Report}

% Optional subtitle
% \setDocSubtitle{A subtitle}

\author{%
Chuck Claver (chair),
David Cabrera (co-Chair),
Rob McKercher (co-chair),
John Andrew,
Diane Hascall,
Patrick Ingraham,
Tony Johnson,
Kristen Metzger,
Austin Roberts,
Matthew Rumore, and 
Jonathan Sick.
}

\setDocRef{SITCOMTN-014}
\setDocUpstreamLocation{\url{https://github.com/lsst-sitcom/sitcomtn-014}}

\date{\vcsDate}

% Optional: name of the document's curator
% \setDocCurator{The Curator of this Document}

\setDocAbstract{%
This document is a report on the recommended future state and organization of Rubin Construction Documentation.  This will constitute the planned technical documentation package delivered from the Construction Project to Operation as part of the criteria for construction completeness.
}

% Change history defined here.
% Order: oldest first.
% Fields: VERSION, DATE, DESCRIPTION, OWNER NAME.
% See LPM-51 for version number policy.
\setDocChangeRecord{%
  \addtohist{1}{YYYY-MM-DD}{Unreleased.}{Chuck Claver}
}


\begin{document}

% Create the title page.
\maketitle
% Frequently for a technote we do not want a title page  uncomment this to remove the title page and changelog.
% use \mkshorttitle to remove the extra pages

% ADD CONTENT HERE
% You can also use the \input command to include several content files.

\section{Introduction}

Put basic description of approach and report structure here.

This report include details on the following topics:

\begin{itemize}

\item DocumentationViews
	\begin{itemize}
	\item ProductView
	\item AccessView
	\item  StorageView
	\end{itemize}
	
\item Implementation Plan
	\begin{itemize}
	\item PrimarySources
	\item DocumentationPortal
	\item DocuSharePath
	\item PDMWorksPath
	\item Resources \& Responsibilities
	\item Schedule
	\item TransitionPlan
	\end{itemize}
	
\item RiskAssessment

\end{itemize}


\section{Documentation Views}
\label{sec:views}

The Documentation Working Group developed four "Documentation~Views" with customized subcategories to functionally describe and organize the documentation for Rubin Observatory Operations.
Each view is represented by a \emph{tree~structure}, a widely used way of representing hierarchical structures in a graphical form.
This proposal includes classical node-link diagrams and tree view outlines that are built with branches of connected nodes, or "leaf nodes,"
starting from "root nodes," or starting nodes that are the highest level in the hierarchy;
however, it may be possible to use different tree structure diagrams depending on the particular use case.
\citep{wiki-tree-diagram-cite}

The four views are:

\begin{itemize}

\item The Product View --- For organizing the ownership of documentation, describing internal systems and providing structure for linking and cross referencing documentation or informational dependencies.
\item The Access View --- For describing the user base and documentation applicable to their use cases.
\item The Storage View --- For determining the normative source of information (e.g., source of truth) and listing documentation locations and/or repositories.
\item The Topic View --- For searching and discovering documentation.

\end{itemize}
 
These views are designed to provide a consistent structure for staff and users to search, reference and retrieve currently available information while considering their expected interests,
and provide a robust and orderly way to include new documentation with the goal of facilitating references to minimize replication.
The following subsections propose how each view's tree is constructed and various examples of use.

It is intended that the implementation of these views will constitute a deliverable from the Construction Project to the Operations Team.
The different managing groups will complete these views/trees to have documentation organized in common way for operations.
It is also important to note that no content currently used in construction or pre-operations will be lost
(see the Rubin Observatory Construction Documentation Inventory, \url{https://sitcomtn-012.lsst.io/}).
This content will be maintained as is for availability,
but not necessary incorporated into the documentation scheme described in this proposal.

	\subsection{Product View}

Describe here the properties of the product View and provide diagrams showing the proposed structure for this view.

The Product View is...

The root nodes consist of the Rubin Observatory Operations Departments.




Directors Office
The Vera C. Rubin Observatory Director’s office is responsible for the overall management of the observatory and the survey as well as fulfilling the mission of the observatory and realizing its vision. The Director’s office includes a Directorate, Administrative Operations, Safety, Communications, and In-Kind Contributions teams.
Observatory Operations
The Chilean-based Rubin Observatory Operations Department is responsible for operating and maintaining the telescope, camera systems, and summit facilities in order to collect the raw imaging and housekeeping data needed by the Legacy Survey of Space and Time. The primary tasks include maintaining the operating facilities, conducting the night-time survey operations, real-time assessment of image quality and observing efficiency, performing the daily calibration, and collecting and analyzing engineering data.
Data Production
The role of the Rubin Observatory Data Production department is to accept data from the Observatory’s telescopes and ancillary systems; to process that data to generate science ready data products; to archive both raw data and derived data products; and, subject to approval from the Science Performance department and the Data Release Board, to make that data available to the scientific community. The Data Production department will develop, maintain and operate the networks, compute and storage hardware, and software that constitutes the Rubin Observatory Data Management System for the duration of the operational period.
System Performance
Rubin Observatory System Performance department is responsible for ensuring that the LSST as a whole is proceeding with the efficiency and fidelity needed to achieve its science requirements at the end of the 10-year survey. This includes the Wide-fast-Deep (WFD) survey and all Special Programs (deep drilling fields and mini-surveys). To meet this goal, the System Performance department will track and optimize the integrated performance of the entire system. This includes the performance of the observatory and the progress of the survey with respect to its science objectives, the ability of the community to access and analyze the data and publish results on the four LSST science pillars at an appropriate rate, the evaluation of strategies for improving the survey strategy, and the development of mitigation strategies together with other relevant departments to minimize the impact of changes in the system performance on the overall LSST science.
Education and Public Outreach
The mission of the Rubin Observatory Education and Public Outreach (EPO) program is to offer accessible and engaging online experiences that provide non-specialists access to, and context for, Rubin Observatory data so anyone can explore the Universe and be part of the discovery process. EPO serves as a website that highlights and contextualizes the scientific power of Rubin Observatory for non-specialists and hosts all online resources.
	\subsection{Access View}
\label{sec:access-view}

The \emph{Access View} is a method of categorizing information into trees serving role-based needs to stakeholders.
Its purpose is to assist users in retrieving and discovering documentation and information applicable to their use cases.
As such, the access trees will be tailored to use case and role-based needs.
The Access View is to be agnostic to the storage location because access is provided through the Documentation Portal so users do not need to know the repository for information.

The following subsections include four recommended trees for the Access View developed by the Documentation Working Group: Operational Access View, Safety Access View, Scientific Access View, and Engineering Access View.
It was natural to develop an Access View tree from a system decomposition, but it is useful for the project to determine if there are more efficient or natural trees, such as activity-based trees.
For each Access View tree, there are many ways and reasons to break the tree into the root-nodes.
With such sprawling associated subjects, it's more crucial to develop a strategy to break these root-nodes and leaf-nodes by considering categories, impacts to users, and how users would want to find associated information.

\subsubsection{Operational Access View}

The Operational Access View tree is intended to help with interfaces of real-time data and visualization, observations and its programming, and system state.
Figure \ref{fig:access-view-operational-example} is an example of the Operational Access View.
It may be natural to have separate leaf-nodes for the Main Telescope and AuxTel, capturing specific differences such as control interfaces.
Another natural break down could be day- and night-time operations, as shown in Figure \ref{fig:access-view-day-night-operations-example}.
Responsible groups should take advantage of commonalities and referential nature of the Documentation Views.
Users include use of equipment control devices, procedures and protocols for operation and maintenance, such as control room staff, viewing assistants.

\begin{figure}[ht]
\centering
\includegraphics[scale=0.7]{access-view-operational-example}
\caption{Example of Operational Access View Tree with Root-nodes.}
\label{fig:access-view-operational-example}
\end{figure}

\begin{figure}[ht]
\centering
\includegraphics[width=\textwidth]{access-view-day-night-operations-example}
\caption{Example of Operational Access View Tree from Daytime and Nighttime Operations.}
\label{fig:access-view-day-night-operations-example}
\end{figure}

\subsubsection{Safety Access View}

The Safety Access View tree focuses on accessing information related directly to safety procedures and protocols that apply to the different systems.
This can help add consistency and verify requirements.
This tree would lead users to general safety procedures, specific safety procedures, emergency procedures of telescope and summit facilities (excludes emergency response).
The example provided by Figure \ref{fig:access-view-safety-example} includes safety and security aspects.
Users include safety, engineering, security, maintenance staffs and management.

\begin{figure}[ht]
\centering
\includegraphics[width=\textwidth]{access-view-safety-example}
\caption{Example of Safety Access View Tree.}
\label{fig:access-view-safety-example}
\end{figure}

\subsubsection{Scientific Access View}

The Scientific Access View tree is for scientific platforms and interfaces that correspond or impact Rubin Observatory science objectives.
It is intended to ease access to the information by providing a clear list of the software and various databases.
An example is provided by Figure \ref{fig:access-view-scientific-example}.
Users include scientific, engineering, and observing staff and management.

\begin{figure}[ht]
\centering
\includegraphics[width=\textwidth]{access-view-scientific-example}
\caption{Example of Scientific Access View Tree.}
\label{fig:access-view-scientific-example}
\end{figure}

\subsubsection{Engineering Access View}

The Engineering Access View tree is for accessing Rubin Observatory technical information.
In addition to the technical documentation, access interfaces and facility-related data would be needed.
An example is provided by Figure \ref{fig:access-view-engineering-example}.
Users include staff and management involved in system performance and system engineering activities.

\begin{figure}[ht]
\centering
\includegraphics[scale=0.7]{access-view-engineering-example}
\caption{Example of Engineering Access View Tree.}
\label{fig:access-view-engineering-example}
\end{figure}

	\subsection{Storage View}
\label{sec:storage-view}

The purpose of the \emph{Storage View} is to capture the official repositories retaining, archiving, organizing, and accessing Rubin Observatory technical documentation in a reliable, consistent manner for users, developers and management.
As a fundamental corollary, the Storage View provides a method for departments and technical groups to define the normative source for information, i.e., a single, definitive source of truth.
With canonical information storage, others can develop associations, links and references to impart information to inter-departmental or external documents or groups, allowing a reduction in replication, reliable information flow, and preservation of informational dependencies.
Unless mandated by the project, the department or responsible group can choose the manner and method of storing information and transposing information to other locations, in conjunction with the stakeholder(s).
The primary users are managers, engineers, specialists, technicians, web development staff, system administrators, and documentation specialist.

To ensure a high degree of reliability, it is crucial staff store all operational and historical information within officially designated locations.
Specifying official repositories will limit the level of effort in the utilization and maintenance of multiple documentation systems, prevent or highly discourage using non-official storage repositories, and reduce risk to storing and accessing operational and historical information.
By limiting the available repositories, effort and resources can focus on best using the official storage locations or modify them in an orderly, consistent manner.

As a shared responsibility between system administrators and departments, all official repositories should be well maintained.
System administrators must ensure access throughout operations (e.g., readily available and backed-up) and a level of expertise must be on-hand for maintenance, issues and general assistance to staff.
Departments and staff must maintain their information and its organization within the repositories.
They should recognize and act when information or utilization becomes outdated or unused as to prevent large amounts of depreciated content, stress on storage and bit rot.
This is especially important to the development, deployment and maintenance of the Documentation Portal architecture (Section \ref{sec:DocPortal}), as leveraging specific use cases and repository's native or project-developed metadata fields are key components.
Additionally, this effort will better suit the project and departments to formalizing a long-term organizational structure of each repository, create customized workflows, transition to the Documentation Views, and implement new or updated documentation.

Non-official storage locations pose unique operational and managerial risks which can lead to information and data loss, increased information security risks, and impact to the project's schedule and budget.
This includes information control (e.g., access, enforced version control, archiving), reliable data recovery, unknown or unplanned risks, and lack of available support (e.g., available labor, expertise).
Platforms may become unsupported or unavailable and the information therein becomes lost or otherwise inaccessible.
This risk has already been realized within the construction project, although of minor impact, as evident from a small number databases (e.g., personal drives, old network drives) that are difficult to access or have lost or inaccessible information. \citedsp{SITCOMTN-012}
Additionally compounding the risk to information access, it becomes increasingly difficult to share information, especially when the repository is ill-defined, uncertainty of what replicated information is the most current, and uncertainty of which repository is the normative source of information.
Without a discrete set of storage locations, it is impossible to guarantee reliable and navigable information or data across multiple platforms throughout operations.

The Storage View tree is similar to a product tree, with the basis of design a result of the \citeds[Construction Documentation Inventory]{SITCOMTN-012}.
This section summarizes some important details from the document, but \citeds{SITCOMTN-012} includes more information such as the inventory for each repository.
Each repository is a root-node, similar to a Level-1 product, and the first set of leaf-nodes captures the location's organizational structure and its metadata structure.
Critically, the Storage View trees should readily indicate what are the normative sources of information so other Documentation Views can reliably refer to the accurate and up-to-date information.
It will be the responsibility of the department or technical group to define the organizational structure, with documentation specialists assisting in developing a consistent approach across the project and sharing lessons learned between groups.
Software developers require the metadata and its structure to develop the Documentation Portal, including the development of additional metadata fields with assistance from system administrators.
Metadata information was not surveyed by the Documentation Working Group; however, metadata fields and structure must be documented for development, changes and implementation of the Documentation Portal.
Recommendations and suggestions by the Documentation Working Group to transition and use the respiratory in operations are provided, including some suggested repository's organizational structure.
The level of effort and schedule impacts are included where appropriate.

\subsubsection{DocuShare}

Xerox\textregistered\ DocuShare\textregistered\ content management system \citep{DocuShare-cite} is the Construction Project's official document repository.
It was selected during the design and development phase to meet the NSF requirement for a document management system.
The Construction Project Office expects DocuShare to be the repository for official versions of management policies, plans and procedures, design documents, safety documentation, hazard analyses, released requirements and interface control documents generated from the SysML model, and project standards, guidelines and templates (this list is not intended to be exhaustive).

Three of DocuShare's advantages are handles, version control and co-location.
Each object has a unique identifier called a handle, which follows the object regardless of versioning or location(s) in the directory structure (called collections).
Each handle has a version history that lists all previously uploaded files for the object in question.
One of those versions is designated as the preferred version, which represents the document's official, approved version and is served by the database when clicking the object's title or from a properly formatted URL shared or hosted outside of DocuShare.
The object's handle does not change when/if a new version of the document is created.
Lastly, objects can appear simultaneously in as many collections as are necessary and/or relevant to the document.
This is facilitated by an object's "locations" property; locations are added as appropriate, and the handle automatically appears in the newly added collection.
The combination of handles, versioning and co-location creates a system where each document is represented by a single record, avoiding duplication and/or version confusion.

The web interface for DocuShare is \href{https://docushare.lsstcorp.org/docushare/dsweb/HomePage}.
Currently, DocuShare contains more than $30,000$ documents in more than $10,000$ collections.
Creation, retention and version control of those documentation classes generally have been well managed; however, the bulk of DocuShare documents likely represents objects created ad hoc by general project staff for specific purposes.
In addition, there are thousands of pieces of work product that may have archival significance but likely will not be useful for Operations.

As a tool heavily used by the project, it is recommended to continue the use of DocuShare into operations.
Operations should use the DocuShare instance currently in use by the construction project with the Archive Server add-on enabled, as the continuity afforded by doing so is important and useful.
Under this model, the construction project content would be moved to the archive server, and the active server would contain only operations and operations-relevant construction content following an agreed-upon directory structure.
Until this transition is completed project-wide, the current location for operational documentation is \href{https://docushare.lsstcorp.org/docushare/dsweb/View/Collection-602}{Collection-602}.
Documentation specialists have been encouraging sub-system staff to organize objects in this collection and store operations information therein.
In the few cases of feedback, this new setting has been positively received, citing the DocuShare structure for construction is difficult to understand and unintuitive for new staff, especially for operations staff.

The suggested DocuShare branch of the Storage View for operations starts with the a collection for each department.
Additional collections can be created based on the project's needs as a whole (e.g., interdepartmental items such as widely used software tools).
The next level of nodes is defined by the system's and/or product's department or owner.
Further lower-level nodes would depend on how the department and owners want to parse the information, e.g., separation by subsystem, separation by topics such as reports, or a combination of the two.
As for construction documentation, it is recommended to have a consistent organizational structure when transferring over to the archive server.
The archiving process should include a determination by the department if the information is useful in operations (determining if it should be archived) and if the information should be queryable in the archive (e.g., raw data).
This is especially topical for vendor documentation and deliverables.

See Section \ref{sec:docushare-path} for a detailed discussion on the reasoning, recommended actions and expected resources to implement and maintain DocuShare in operations.
A more detailed analysis of arguments and considerations for this proposal's recommendation is available in DocuShare Options Trade Study for the Documentation Working Group. \citedsp{Document-36788}

\subsubsection{LSST the Docs (www.lsst.io)}

LSST the Docs (LTD), also known by its URL "lsst.io", is a documentation hosting platform built and operated by the SQuaRE team within the Data Management group.
LTD hosts \emph{versioned static websites}, meaning any website built from HTML, CSS and JavaScript that doesn't need an active server to render content (as opposed to say Confluence, DocuShare, or Drupal websites).
Static websites are a natural fit for documentation projects that originate from repositories hosted by \href{https://www.github.com}{GitHub} \citep{GitHub-cite}, so LTD is uniquely developed to be built around versioned documentation in GitHub.
\emph{lsst.io} is one such deployment used as a hosting domain for Rubin Observatory, where all subdomains of lsst.io are independent documentation projects.
The technical motivation and design of LTD are documented in \href{https://sqr-006.lsst.io}{SQR-006: The LSST the Docs Platform for Continuous Documentation Delivery}. \citeds{SQR-006}
The key technical features of LTD are:

\begin{itemize}
	\item high reliability, scaling, and security: documentation is hosted in Amazon S3 and served through the Fastly content distribution network.
	We don't operate any servers that receive traffic from users;
	\item versioned documentation; and
	\item flexibility to host any type of static website.
\end{itemize}

Using LTD documents provides a simple use case with additional features developed by the SQuaRE team.
The root URL for a documentation project hosts the "default" version, which has a configurable meaning for each project (such as software release versions, temporary collaborative drafts, or an active version in DocuShare).
Users can also browse other versions of the documentation through the "/v/" dashboard pages (for example \url{https://www.lsst.io/v/}).

LTD hosts two types of documentation projects: \emph{guides} and \emph{documents}.
Guides are multi-page websites convenient for user interaction and navigation (e.g., \href{developer.lsst.io}{Data Management Developer Guide}, T\&S software guides at \url{https://obs-controls.lsst.io}).
Documents are "single-page" artifacts, analogous to documents that might be found in DocuShare, and they are sometimes referred to as \emph{technical notes} (shortened to "technote" or an appended "TN") --- see \href{https://sqr-000.lsst.io}{SQR-000: The LSST DM Technical Note Publishing Platform} for the motivation to create technical notes. \citedsp{SQR-000}
Though not required, guides are generally authored using an open-source tools using themes that is maintained by the SQuaRE team via documenteer \cite{documenteer-cite} (see \url{https://documenteer.lsst.io/} for corresponding guide).
Besides documentation tied to specific software projects or services, guides can also collect procedures for teams, see the DM Developer Guide (\url{https://developer.lsst.io}) or the Observatory Operations Documentation (\url{https://obs-ops.lsst.io}).

DM has embraced the use of LTD by hosting a guide on lsst.io for every software project or service.
Further, DM has developed most of its change-controlled documents (LDM) on its lsst.io site to take advantage of the sophisticated collaboration features that GitHub offers (for an example, see \url{https://ldm-151.lsst.io}).
Change-controlled documents are submitted to DocuShare for archival once approved using a release process mediated through GitHub, Jira, and the relevant control board.
LTD is currently hosting documents from the DMTR, LDM, LPM, LSE, and SCTR document series (note that this includes test and verification reports).

A prime example of the LTD and GitHub tools used by the project for operations activities is the T\&S Commandable Service Abstraction Layer (SAL) Component (CSC) XML package and user guides, as summarized by \url{https://ts-xml.lsst.io}.
The CSC data objects are defined in XML for SAL to consume and produce language-specific libraries that enable communication over the Data Distribution System (DDS) network.
These XML files are critical for defining the configuration and interaction of the systems within the Summit Facility observing environment.

A unique example is the homepage for the LTD documentation platform, \url{https://www.lsst.io}, which serves as a portal for LTD indexed documentation for searches and faceted browsing capabilities. \citep{lsst.io-cite}
Users can search across metadata and full text (this feature is powered by the commercial service Algolia \citep{Algolia-cite} in conjunction with a scraper bot built by SQuaRE) or browse through curated collections.
The site itself is built with React/Gatsby.js (\url{https://github.com/lsst-sqre/www_lsst_io}), the search database is SaaS \citep{SaaS-cite}), and the bot that indexes content into the search database is called Ook (\url{https://github.com/lsst-sqre/ook}).
It is still in development and the current status is documented at \url{https://www.lsst.io/about/}.

As a customized set of tools that is heavily used by the project, it is recommended to continue the use of LTD and its software system into operations.
The project should use documents and technotes to capture information that is static, rarely changed or serving a temporary need (e.g., proposals, documenting proof or principles, status updates);
whereas, guides should be used for information that is actively updated with current information (e.g., troubleshooting guides, procedures not under change control).
Neither LTD documentation type should be used for documents under change control albeit one can link to the change-controlled document within a document or guide.
Further, as the LTD software system, it's additional utilities and \url{https://www.lsst.io} are designed and built by the SQuaRE team, there is considerable opportunity to leverage the expertise and experience already developed to implement a documentation portal for other platforms (see Section \ref{sec:DocPortal}).

The LTD branch high-level nodes of the Storage View could be by document type (i.e., guides or documents).
Following the department/owner or specific technote series (e.g., RTN, SITCOMTN) could be the next level of nodes.

\subsubsection{Confluence and Jira}
\label{confluence-jira-storage}

Confluence \citep{Confluence-cite} and Jira \citep{Jira-cite} are part of the Atlassian Corperation suite of tools used by the construction and operations projects for many purposes.
They are collaborative tools where teams/groups can document, share and develop information.
Confluence is organized into \emph{spaces}, each with a varying number of pages and sub-pages.
Jira is organized into \emph{projects}, each of which tracks a list of enumerated tickets or issues.
Both include a large variety of features, tools and extendable add-ons to manipulate or organize the information.
Many of these spaces and projects are very specialized, with some set up for personal use.

On the construction project, there are two instances of each software tool, requiring different credentials --- \url{https://confluence.lsstcorp.org} and \url{https://jira.lsstcorp.org} is for the NSF MREFC effort hosted by LSSTC out of Tucson, Arizona; and, \url{https://confluence.slac.stanford.edu/} and \url{https://jira.slac.stanford.edu/} is for the DOE MIE effort for LSSTCam hosted by SLAC National Accelerator Laboratory out of Stanford, California.
Rubin Observatory operations and pre-operations staff currently use the LSSTC tools.

Currently, the LSSTC Jira instance includes the following normative source of information:
(1) verification elements, plans, cycles, cases, results, etc.,
(2) construction-related risks, opportunities and mitigations,
(3) the Failure Reporting Analysis and Corrective Action System (FRACAS) for failures and corrective actions,
and (4) hazard mitigation verification.
However, it is required operations include risks, opportunities and mitigations in the NOIRLab managed risk management software tool, Alcea Tracking Solutions software \url{https://noirlab.alceatech.com/saml2/sso}.
The project must determine if the normative source for information on risk, opportunities and mitigations will continue to be Jira or moved to the NOIRLab risk management tool.
If Jira is the normative source of information for these items, the project must develop a method to sync information to the NOIRLab tool.

As tools that are heavily used by the project, it is recommended to continue the use of Confluence and Jira into operations.
With a wiki like Confluence which that integrates well with Jira, the two are especially convenient for rapidly developing new ideas, taking and storing meeting minutes, collecting information in interactive tables, and drafting outlines for future documentation, all while tracking the tasks and being able to actively report on status.
Other powerful features include simultaneous editing, native sharing options internal to Confluence/Jira or external such as email, etc.
As operational-based information takes form, departments and technical groups should consider the spaces and projects to prevent having a large number of unused or old areas, as seen with the current Confluence and Jira instances.
In the case of Confluence, project staff must be fastidious in moving content needed on a long-term basis into other official storage locations at the earliest appropriate stage (e.g., DocuShare, lsst.io).
Users must understand the limitations of information and provided guidance as to how and when information should be moved from Confluence or Jira to another storage location.

The Confluence branch of the Storage View would begin with the spaces.
Most spaces have a relatively small number of top-level pages which are tailored to a specific need (e.g., separation of subsystems), and these top-level pages can naturally be the next level of nodes.
The Jira branch would begin with the projects, but the next level nodes are not as apparent and should be determined by the owner or responsible group.
There are many ways to create the next level nodes for Jira; for example, most projects have a natural breakdown of structure used to organize relationships between tickets (e.g., epics).
Lower-level nodes will depend on use; they could include items such as meeting notes or customized dashboards.
The NOIRLab risk tool should also have a branch in the Storage View.

\subsubsection{Engineering models in Solidworks Product Data Management (PDM)}

Solidworks Product Data Management (PDM) Professional \citep{PDM-cite} is the official computer-aided design (CAD) model repository for the Telescope and Site construction group, including vendor documentation and deliverables.
It uses a check-out / check-in system to allow configuration management of the design.
Each check-in produces a new version of the part or assembly.
Earlier versions can be accessed if needed to compare designs or revert to an earlier design.
A workflow feature allows the designs to go through a review process until the design is approved and locked from further changes.
A revision process is also included in the workflow to allow for changes to the designs if needed after final approval.
The software allows for vault replication at multiple sites and the project currently has a server operating in Tucson, Arizona and Chile to support CAD users at multiple sites.
Solidworks PDM is the normative source of information for system decomposition.

The current configuration of the PDM vault contains top-level access to baseline design data and interface control documents (ICDs) (drawings) along with as-designed vendor subsystems.
This includes the original baseline design, early designs and further development throughout construction.
In addition, a series of design and drafting standards is also stored in the PDM system.
Two servers are hosted: the main server located in Tucson, Arizona, and a replicated server in La Serena, Chile to allow faster file access.

In operations, as-built design information will be needed to help with logistics planning for maintenance and future design upgrades.
No information has been deleted or archived at this time.
The Documentation Working Group recommends the vault is reorganize so that legacy data is archived for access but is not easily mistaken for as-built design data.
This will require a full-time resource for 3 to 6 months.
The specific structure for the branch of the Storage View is highly dependent on this reorganization.
For example, the high-level nodes could be by subsystem or physical location, lifecycle of particular drawings (e.g., as-designed, as-built), or use within specific states of the telescope (e.g., on-sky imaging, preventive maintenance shutdown).

See Section \ref{sec:pdm-path} for a detailed discussion on the reasoning, recommended actions and expected resources to implement and maintain Solidworks PDM in operations.

\subsubsection{MagicDraw}

MagicDraw \citep{MagicDraw-cite} is a tool used to maintain a model of the Rubin Observatory system by creating a relational database between system elements.
There are a number of elements capture in the MagicDraw database defined as the normative source of information, with the information synced and/or exported to other repositories.
As the normative source of information, it will be up-to-date when given over from construction and continue active use within operations.
Details on how Rubin Observatory uses MagicDraw are included in the user guides collected on the following Confluence page: \url{https://confluence.lsstcorp.org/display/SYSENG/MagicDraw+LSST+Users+Guide}.

\begin{itemize}
	\item \textbf{Hazard Analysis} --- normative source --- synced to Jira for verification tasks.
	\item \textbf{Failure Mode and Effects Analysis (FMEA)} --- normative source --- known to be incomplete.
	\item \textbf{Requirements} --- normative source --- exports to DocuShare.
	\item \textbf{Verification Elements} --- normative source --- synced to/from Jira.
	\item \textbf{Verification Plans/Cycles/Cases} --- synced to/from Jira via Syndeia\texttrademark\ \citep{syndeia-cite}.
	See Section \ref{confluence-jira-storage} for normative source).
	\item \textbf{SAL Commands, Events, Telemetry} --- imported from CSC XML.
	\item \textbf{Operations Concepts} --- source of truth.
	\item \textbf{System-level State Machine} --- source of truth.
	\item \textbf{Interlocks} --- modeled from source material.
	\item \textbf{Structural Decomposition} --- will be synced from Solidworks in the future.
\end{itemize}

It is recommended to continue the use of MagicDraw into operations.
The user guides should be relocated into a more appropriate storage location, such as lsst.io.
The content in MagicDraw should be reviewed to create the tree for the Storage View that can readily reflect normative sources of information and how/what references or depends on this information.
The effort required to complete the FMEA information will depend on the level of completeness when handed over to operations.
Effort to sync the structural decomposition with Solidworks will be evaluated when the system information is more complete.

\subsubsection{Euporie}

Euporie is a network drive on a server managed by the construction project; and, it is accessible with Rubin Observatory credentials through VPN at \url{smb://euporie}.
The drive contains a number of personal directories and a directory named "TS-Deliverables" managed by the Telescope and Site group.
Stored in subdirectories of TS-Deliverables are vendor-supplied documentation as contract deliverables (design documentation, construction drawings, manuals and other miscellaneous information), with each subdirectory.

It is recommended to discontinue use of Euporie for operations.
A review of the content of personal and shared directories is recommended to determine what information should be relocated on a case-by-case basis.
Additionally, a determination should be made if information should be archived as construction-related or moved to a more active storage location for operations staff access.
After this is complete and access will continue for purpose unrelated to operations, users must understand the limitations of information (namely for long-term operational activities) and provided guidance as to how and when information should be moved from Euporie to another storage location.

\subsubsection{GitHub}

GitHub\texttrademark\ \citep{GitHub-cite} is used by the project for software and documentation collaboration, storage, version control via git\texttrademark\ \citep{git-cite} and deployment.
GitHub is primarily utilized via \emph{repositories}, or repos.
Repositories are owned by individual users or an \emph{organization}, and organizations can include \emph{teams} for additional granularity.

The Rubin Observatory construction and operations projects use a large number of GitHub repositories, organizations and teams, primarily to ease access control.
The main project GitHub organization is \url{https://github.com/lsst}; however, not all operations software is found in this organization.
On the construction project, each subsystem on the construction project generally has its own GitHub organization, though in some cases subsystems have additional GitHub organizations for specific teams or projects.
For example, the Science Platform software is contained in \url{https://github.com/lsst-sqre}, Telescope \& Site software in \url{https://github.com/lsst-ts}, EPO software in \url{https://github.com/lsst-epo} and Camera software in several organizations.

As a heavily used tool by the project, it is recommended to continue the use of GitHub via git into operations.
Effort by construction, pre-operations and operations teams already implemented infrastructure for utilization and the GitHub collaboration structure, including basic structure for operations.
There are a large number of repositories containing the normative source of information, and the project should ensure these repositories are clearly indicated as such and what respective information is the normative source.

\subsubsection{Drupal}

Drupal \citep{drupal-cite} is an open source web content management framework used by the project for its websites’ back-end, including \url{https://www.lsst.org}, \url{https://project.lsst.org} and sites created to facilitate project meetings and reviews.
To avoid the complication of having to make multiple site updates when content changes, these documents of each of these sites are served by hyperlinks that pull files from whatever repositories contain their normative sources of information.
While it is possible to upload discrete files to a specific site’s server location, by policy and standard the project eschews doing so.

The major exception is regarding project-level meetings and reviews sites, such as the Project and Community Workshop (PCW) or agency reviews.
Their site directories contain document and presentation files in order to preserve the content presented at the time.
At the conclusion of the event, these files are uploaded to DocuShare in a collection specific to the event.
Documents such as policies, requirements, and design documents are uploaded as a ZIP file to preserve a snap-shot while preventing replication and multiple handles that may cause confusion with the normative source of information.
The method described for project-level meetings and reviews may not be implemented by all subsystems on the construction project, and the project must consider if and how to archive this information.

The project must consider if and how its appropriate to continue using Drupal for web content management into operations.
If used in operations, it must be clear how and where normative sources of information are used (e.g., DocuShare).
It must be determined the level of change control appropriate for websites, as well as files in site directories or hyperlinked files without change control.

\subsubsection{LSSTCam information}

There are multiple storage locations for information, documents and data produced during the MIE effort for LSSTCam:



\subsubsection{Engineering Facility Database (EFD)}



\subsubsection{Primavera P6}



\subsubsection{Verification reports}



\subsubsection{CMMS}



\subsubsection{NOIRLab Risk Tool}



	\subsection{Topic View}

The \emph{Topic View} is an approach to categorize information into document types.
Its purpose is to allow users the ability to find information of a specific form within a system.
It first requires a defined decomposition of systems, subsystems and components.
With that information, documentation can be categorized into the following:

\begin{itemize}
\item design documents
\item requirement documents
\item technical manuals (e.g., operation or maintenance)
\item operational documents (e.g., procedures)
\item data performance
\item evaluation processes
\item maintenance reports
\item safety and hazard mitigations
\item access and software
\end{itemize}



\section{Transition and implementation planning in operations}
\label{sec:implementation}

the need of defining the workflow of the documentation and data produced leads to create an official group of determined actions to officially save the operational documentation, classify it, separate it by the specific Department, System, Subsystem, facility, type, and application. 

Add here an overview of the proposed implementation plan including the following specific areas of implementation:

\begin{itemize}

\item Primary Sources
\item Documentation Portal
\item DocuShare
\item PDM Works
\item Resources and Responsibilities
\item Schedule
\item Transition Workflow
\item Special Topics (e.g., Bilingual)

\end{itemize}

\begin{figure}[t]
\caption{Temporary Caption.}
\centering
\includegraphics[width=\textwidth]{operations-documentation-workflow-temp}
\label{fig:operations-documentation-workflow}
\end{figure}

	\subsection{Primary Documentation Sources}
\label{sec:primary-sources}

Describe here the proposed location of the primary documentation sources.  These are a subset of the identified source listed in SITCOMTN-012.


	\subsection{The Documentation Portal}

Describe the lsst.io like portal implementation plan.
	\subsection{The Path Forward with DocuShare}

The working group recommends that Rubin continue to use DocuShare in Operations; further, we recommend that Operations use the DocuShare instance
currently in use by Construction with the Archive Server add-on enabled. The working group believes the continuity afforded by doing so is important and useful.
DocuShare’s version control, version history, permissions and co-location capabilities are valuable tools for effective document management and should be 
maintained. Additionally, continuing DocuShare use will ease the transition from Construction to Operations by providing a single shared platform, thereby 
avoiding compatibility issues, eliminating the need to migrate documents, preventing loss of documentation and avoiding confusion by retaining legacy handles 
and references. Enabling the Archive Server will allow Operations to structure and manage DocuShare according to its needs while retaining access to all 
documentation accumulated during Construction. The working group believes the recommendation’s benefits outweigh any required new and/or continuing costs 
and labor resources.

The recommendation is consistent with NOIRLab document management, which has selected DocuShare as the lab’s document repository. By also using 
DocuShare, Rubin Operations’ repository will be compatible with the knowledge and skills held by the business unit responsible for NOIRLab document 
management. While Rubin Operations has the option to use NOIRLab’s DocuShare, the working group believes that to be less advantageous than retaining 
Rubin Construction’s DocuShare instance. Switching to NOIRLab’s DocuShare would result in loss of legacy document handles and would require labor-
intensive and time-consuming document migration. Continuing to use Rubin DocuShare will retain documents’ version histories and allow persistence of handles 
for foundation documents, many of which are known within the project and community as much by their handles as they are by their titles. Lastly, each instance 
of DocuShare has a $2,000,000$ document limit. If the Construction project DocuShare instance is retained, the entire document limit is available to Rubin. If 
NOIRLab’s DocuShare instance is used, Rubin would have to share the document limit with other lab units and programs.

However, the working group recognizes that simply continuing with the Rubin DocuShare as-is fails to address long-standing issues of clutter, inadequate 
metadata, and poor classification; therefore, we recommend enabling the Archive Server add-on. The add-on provides a second server into which documents 
whose lifecycles have ended can be sequestered, freeing the active server to be structured and managed according to best practices and Operations’ needs. 
While archived documents no longer reside in the active server and are not represented in its directory structure, they are viewable and searchable and can be 
restored to the active server if necessary. The two servers are connected and can be interacted with using a single user interface. Under this model, 
Construction project content would be moved to the archive server, and the active server would contain only Operations and Operations-relevant Construction 
content following an agreed-upon directory structure.

Further, to mitigate the risk of replicating the clutter and less-than-optimal classification of Construction-era DocuShare, the working group recommends Rubin 
adopt a more formal workflow for how and by whom DocuShare content is created, formatted, and updated. This workflow would involve something similar to 
NOIRLab’s steward concept. NOIRLab’s document management plan expects each business unit or program to designate a person or persons to lead the 
group’s documentation activities in DocuShare. These stewards are to be highly trained so they can perform DocuShare functions, provide guidance to others, 
enforce standards, and correct errors. The idea is to have fewer but better-trained users with full DocuShare permissions and functionality. While this will result in 
a less distributed workload, consistency and quality will improve, leading to greater tool utility. The Operations departments should follow a similar approach, and 
as much as possible, the workflow should occur within DocuShare.

Retaining the current Rubin DocuShare instance and enabling the Archive Server add-on requires both new and continuing costs and labor. The calendar year 
2020 renewal cost for Rubin’s current DocuShare instance was \$22,000, and Rubin's IT resources are expended to maintain the service, which is hosted on a 
physical server in Tucson. The archive server add-on carries a one-time cost of \$7,500 plus an additional \$1,400 per year for support, according to a December 
2020 quote from Xerox. Additionally, IT resources will be needed to configure the add-on, and some training likely will be required. Despite these considerations, 
the working group believes the benefits outweigh the costs, and the required resources are less burdensome than those that would be needed to adopt another 
repository.

A more detailed analysis of pro and cons is available in DocuShare Options Trade Study for the Documentation Working Group \href{https://docushare.lsst.org/docushare/dsweb/Get/Document-36788/DocuShareOptionsTradeStudy.pdf}{(Document-36788)}.

	\subsection{The Path Forward with PDWM Works}

Describe the maintenance plan of the existing CAD models in PWDM works.
	\subsection{Required Resources \& Subsystem Responsibilities}

Description of the estimated resources need to cary out this implementation plan and what the roles and responsibilities are for each subsystem in this context.
	\subsection{schedule}

Outline the schedule needed for implementation to meet the objective delivering a coherent tachnical document packed at the end of the Project Construction effort.
	
\section{Transition Plan and Workflow}

Describe the process for transition from the current state to the future state.

	
\section{Risk Assessment}

Describe the risk exposure if various part of the whole of this implmentation is not conducted.
	


\appendix
% Include all the relevant bib files.
% https://lsst-texmf.lsst.io/lsstdoc.html#bibliographies
\section{References} \label{sec:bib}
\renewcommand{\refname}{} % Suppress default Bibliography section
\bibliography{local,lsst,lsst-dm,refs_ads,refs,books}

% Make sure lsst-texmf/bin/generateAcronyms.py is in your path
\section{Acronyms} \label{sec:acronyms}
\addtocounter{table}{-1}
\begin{longtable}{p{0.145\textwidth}p{0.8\textwidth}}\hline
\textbf{Acronym} & \textbf{Description}  \\\hline

API & Application Programming Interface \\\hline
CMMS & Computerized Maintenance Management System \\\hline
CSC & Commandable SAL Component \\\hline
ComCam & The commissioning camera is a single-raft, 9-CCD camera that will be installed in LSST during commissioning, before the final camera is ready. \\\hline
DDS & Data Distribution System \\\hline
DM & Data Management \\\hline
DMTR & DM Test Report \\\hline
DOE & Department of Energy \\\hline
DP & Data Production \\\hline
EFD & Engineering and Facility Database \\\hline
EPO & Education and Public Outreach \\\hline
FMEA & failure modes and effect analysis \\\hline
FRACAS & Failure Reporting Analysis and Corrective Action System \\\hline
GIS & Global Interlock System \\\hline
HTML & HyperText Markup Language \\\hline
HTTP & HyperText Transfer Protocol \\\hline
HVAC & Heating, Ventilation, and Air Conditioning \\\hline
IMS & Integrated Master Schedule \\\hline
IT & Information Technology \\\hline
L1 & Lens 1 \\\hline
L2 & Lens 2 \\\hline
L3 & Lens 3 \\\hline
LDM & LSST Data Management (Document Handle) \\\hline
LOVE & LSST Operations Visualization Environment \\\hline
LPM & LSST Project Management (Document Handle) \\\hline
LSE & LSST Systems Engineering (Document Handle) \\\hline
LSST & Legacy Survey of Space and Time (formerly Large Synoptic Survey Telescope) \\\hline
LSSTC & LSST Corporation \\\hline
LaTeX & (Leslie) Lamport TeX (document markup language and document preparation system) \\\hline
M1 & primary mirror \\\hline
M2 & Secondary Mirror \\\hline
M3 & tertiary mirror \\\hline
MIE & Major Item of Equipment \\\hline
MREFC & Major Research Equipment and Facility Construction \\\hline
NSF & National Science Foundation \\\hline
PCW & Project Community Workshop \\\hline
PDF & Portable Document Format \\\hline
RTN & Rubin Technical Note \\\hline
S3 & (Amazon) Simple Storage Service  \\\hline
SAL & Service Abstraction Layer \\\hline
SE & System Engineering \\\hline
SLAC & SLAC National Accelerator Laboratory \\\hline
SP & Story Point \\\hline
SQR & SQuARE document handle \\\hline
SQuaRE & Science Quality and Reliability Engineering \\\hline
SaaS & Software as a Service \\\hline
T\&S & Telescope and Site \\\hline
TMA & Telescope Mount Assembly \\\hline
TS & Test Specification \\\hline
URL & Universal Resource Locator \\\hline
VPN & virtual private network \\\hline
WBS & Work Breakdown Structure \\\hline
WFD & Wide Fast Deep \\\hline
XML & eXtensible Markup Language \\\hline
\end{longtable}

% If you want glossary uncomment below -- comment out the two lines above
%\printglossaries





\end{document}
