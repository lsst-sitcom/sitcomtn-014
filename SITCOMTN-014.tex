\documentclass[SE,authoryear,toc]{lsstdoc}
% lsstdoc documentation: https://lsst-texmf.lsst.io/lsstdoc.html
\input{meta}

% Package imports go here.

% Local commands go here.

%If you want glossaries
%\input{aglossary.tex}
%\makeglossaries

\title{Project Documentation Future State Report}

% Optional subtitle
% \setDocSubtitle{A subtitle}

\author{%
Chuck Claver
}

\setDocRef{SITCOMTN-014}
\setDocUpstreamLocation{\url{https://github.com/lsst-sitcom/sitcomtn-014}}

\date{\vcsDate}

% Optional: name of the document's curator
% \setDocCurator{The Curator of this Document}

\setDocAbstract{%
This document is a report on the recommended future state and organization of Rubin Construction Documentation.  This will constitute the planned technical documentation package delivered from the Construction Project to Operation as part of the criteria for construction completeness.
}

% Change history defined here.
% Order: oldest first.
% Fields: VERSION, DATE, DESCRIPTION, OWNER NAME.
% See LPM-51 for version number policy.
\setDocChangeRecord{%
  \addtohist{1}{YYYY-MM-DD}{Unreleased.}{Chuck Claver}
}


\begin{document}

% Create the title page.
\maketitle
% Frequently for a technote we do not want a title page  uncomment this to remove the title page and changelog.
% use \mkshorttitle to remove the extra pages

% ADD CONTENT HERE
% You can also use the \input command to include several content files.

\appendix
% Include all the relevant bib files.
% https://lsst-texmf.lsst.io/lsstdoc.html#bibliographies
\section{References} \label{sec:bib}
\renewcommand{\refname}{} % Suppress default Bibliography section
\bibliography{local,lsst,lsst-dm,refs_ads,refs,books}

% Make sure lsst-texmf/bin/generateAcronyms.py is in your path
\section{Acronyms} \label{sec:acronyms}
\addtocounter{table}{-1}
\begin{longtable}{p{0.145\textwidth}p{0.8\textwidth}}\hline
\textbf{Acronym} & \textbf{Description}  \\\hline

API & Application Programming Interface \\\hline
HTTP & HyperText Transfer Protocol \\\hline
IT & Information Technology \\\hline
LSE & LSST Systems Engineering (Document Handle) \\\hline
LaTeX & (Leslie) Lamport TeX (document markup language and document preparation system) \\\hline
PDF & Portable Document Format \\\hline
SE & System Engineering \\\hline
UI & User Interface \\\hline
VPN & virtual private network \\\hline
\end{longtable}

% If you want glossary uncomment below -- comment out the two lines above
%\printglossaries





\end{document}
