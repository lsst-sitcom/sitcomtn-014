\subsection{Storage View}
\label{sec:storage-view}

The purpose of the \emph{Storage View} is to capture the official repositories retaining, archiving, organizing, and accessing Rubin Observatory technical documentation in a reliable, consistent manner for users, developers and management.
The Storage View provides a method for departments and technical groups to define locations for respective normative sources of information.
With canonical information storage, others can develop associations, links and references to impart information to inter-departmental or external documents or groups, allowing a reduction in replication, reliable information flow, and preservation of informational dependencies.
Unless mandated by the project, the department or responsible group, in conjunction with the stakeholder(s), can choose the manner and method of storing information and transposing information to other locations.
The primary users are managers, engineers, specialists, technicians, web development staff, system administrators, and documentation specialist.

To ensure a high degree of reliability, it is crucial staff store all operational and historical information within officially designated locations.
Specifying official repositories will limit the level of effort in the utilization and maintenance of multiple documentation systems, prevent or highly discourage using non-official storage repositories, and reduce risk to storing and accessing operational and historical information.
By limiting the available repositories, effort and resources can focus on best using the official storage locations or modify them in an orderly, consistent manner.

As a shared responsibility between system administrators and departments, all official repositories should be well maintained.
System administrators must ensure access throughout operations (e.g., readily available and backed-up) and a level of expertise must be on-hand for maintenance, issues and general assistance to staff.
Departments and staff must maintain their information and its organization within the repositories.
They should recognize and act when information or utilization becomes outdated or unused as to prevent large amounts of depreciated content, stress on storage and bit rot.
This is especially important to the development, deployment and maintenance of the Rubin Documentation Portal architecture (Section \ref{sec:DocPortal}), as leveraging specific use cases and repository's native or project-developed metadata fields are key components.
Additionally, this effort will better suit the project and departments by formalizing a long-term organizational structure of each repository, create customized workflows, transition to the Documentation Views, and implement new or updated documentation.

Non-official storage locations pose unique operational and managerial risks which can lead to information and data loss, increased information security risks, and impact to the project's schedule and budget.
This includes information control (e.g., access, enforced version control, archiving), reliable data recovery, unknown or unplanned risks, and lack of available support (e.g., available labor, expertise).
Platforms may become unsupported or unavailable and the information therein becomes lost or otherwise inaccessible.
This risk has already been realized within the construction project, although of minor impact, as evident from a small number databases (e.g., personal drives, old network drives) that are difficult to access or have lost or inaccessible information. \citedsp{SITCOMTN-012}
Additionally compounding the risk to information access, it becomes increasingly difficult to share information, especially when the repository is ill-defined, uncertainty of what replicated information is the most current, and uncertainty of which repository is the normative source of information.
Without a discrete set of storage locations, it is impossible to guarantee reliable and navigable information or data across multiple platforms throughout operations.

A Storage View tree is constructed a root-node for a repository and the first set of leaf-nodes capturing the repository's organizational structure and metadata structure.
Critically, each Storage View tree should readily indicate what are the normative sources of information so it's accurately referable within other Documentation Views.
It will be the responsibility of the department or technical group to define the organizational structure.
Software developers require the metadata and repository's organizational structure to design and implement the Rubin Documentation Portal, including the development of additional metadata fields with assistance from system administrators.

This section summarizes some important details from \citeds{SITCOMTN-012}; however, that document includes detailed information, such as the inventory for each repository.
Recommendations from the Documentation Working Group for the repositories are provided, including transitional plans, future in operations, and some suggested repository's organizational structure.

\subsubsection{DocuShare}

Xerox\textregistered\ DocuShare\textregistered\ content management system \citep{DocuShare-cite} is the Construction Project's official document repository.
It was selected during the design and development phase to meet the NSF requirement for a document management system.
The Construction Project Office expects DocuShare to be the repository for official versions of management policies, plans and procedures, design documents, safety documentation, hazard analyses, released requirements and interface control documents generated from the SysML model, and project standards, guidelines and templates.
This list is not intended to be exhaustive.

As a tool heavily used by the project, it is recommended to continue the use of DocuShare into operations.
Operations should use the DocuShare instance currently in use by the construction project with the Archive Server add-on enabled, as the continuity afforded by doing so is important and useful.
Under this model, the construction project content would be moved to the archive server, and the active server would contain only operations and operations-relevant construction content following an agreed-upon directory structure.
Until this transition is completed project-wide, the current location for operational documentation is \href{https://docushare.lsstcorp.org/docushare/dsweb/View/Collection-602}{Collection-602}.

The suggested DocuShare branch of the Storage View for operations starts with the a collection for each department.
Additional collections can be created based on the project's needs as a whole (e.g., interdepartmental items such as widely used software tools).
The next level of nodes is defined by the system's and/or product's department or owner.
Further lower-level nodes would depend on how the department and owners want to parse the information, e.g., separation by subsystem, separation by topics such as reports, or a combination of the two.
As for construction documentation, it is recommended to have a consistent organizational structure when transferring over to the archive server.
The archiving process should include a determination by the department if the information is useful in operations (determining if it should be archived) and if the information should be queryable in the archive (e.g., raw data).
This is especially topical for vendor documentation and deliverables.

A more detailed analysis of arguments and considerations for this proposal's recommendation is available in DocuShare Options Trade Study for the Documentation Working Group. \citedsp{Document-36788}

\subsubsection{LSST the Docs (www.lsst.io)}

LSST the Docs (LTD), also known by its URL "lsst.io", is a documentation hosting platform built and operated by the SQuaRE team within the Data Management group.
LTD hosts \emph{versioned static websites}, meaning any website built from HTML, CSS and JavaScript that doesn't need an active server to render content (as opposed to say Confluence, DocuShare, or Drupal websites).
Static websites are a natural fit for documentation projects that originate from repositories hosted by \href{https://www.github.com}{GitHub} \citep{GitHub-cite}, so LTD is uniquely developed to be built around versioned documentation in GitHub.
\emph{lsst.io} is one such deployment used as a hosting domain for Rubin Observatory, where all subdomains of lsst.io are independent documentation projects.
The technical motivation and design of LTD are documented in \href{https://sqr-006.lsst.io}{SQR-006: The LSST the Docs Platform for Continuous Documentation Delivery}. \citeds{SQR-006}
The key technical features of LTD are:

\begin{itemize}
	\item high reliability, scaling, and security: documentation is hosted in Amazon S3 and served through the Fastly content distribution network.
	We don't operate any servers that receive traffic from users;
	\item versioned documentation; and
	\item flexibility to host any type of static website.
\end{itemize}

Using LTD documents provides a simple use case with additional features developed by the SQuaRE team.
The root URL for a documentation project hosts the "default" version, which has a configurable meaning for each project (such as software release versions, temporary collaborative drafts, or an active version in DocuShare).
Users can also browse other versions of the documentation through the "/v/" dashboard pages (for example \url{https://www.lsst.io/v/}).

LTD hosts two types of documentation projects: \emph{guides} and \emph{documents}.
Guides are multi-page websites convenient for user interaction and navigation (e.g., \href{developer.lsst.io}{Data Management Developer Guide}, T\&S software guides at \url{https://obs-controls.lsst.io}).
Documents are "single-page" artifacts, analogous to documents that might be found in DocuShare, and they are sometimes referred to as \emph{technical notes} (shortened to "technote" or an appended "TN") --- see \href{https://sqr-000.lsst.io}{SQR-000: The LSST DM Technical Note Publishing Platform} for the motivation to create technical notes. \citedsp{SQR-000}

A unique example is the homepage for the LTD documentation platform, \url{https://www.lsst.io}, which serves as a portal for LTD indexed documentation for searches and faceted browsing capabilities. \citep{lsst.io-cite}
Users can search across metadata and full text (this feature is powered by the commercial service Algolia \citep{Algolia-cite} in conjunction with a scraper bot built by SQuaRE) or browse through curated collections.
The site itself is built with React/Gatsby.js (\url{https://github.com/lsst-sqre/www_lsst_io}), the search database is SaaS \citep{SaaS-cite}), and the bot that indexes content into the search database is called Ook (\url{https://github.com/lsst-sqre/ook}).
It is still in development and the current status is documented at \url{https://www.lsst.io/about/}.

As a customized set of tools that is heavily used by the project, it is recommended to continue the use of LTD and its software system into operations.
The project should use documents and technotes to capture information that is static, rarely changed or serving a temporary need (e.g., proposals, documenting proof of principles, status updates);
whereas, guides should be used for information that is actively updated with current information (e.g., troubleshooting guides, procedures not under change control).
Neither LTD documentation type should be used for documents under change control albeit one can link to the change-controlled document within a document or guide.
Further, as the LTD software system, it's additional utilities and \url{https://www.lsst.io} are designed and built by the SQuaRE team, there is considerable opportunity to leverage the expertise and experience already developed to implement a documentation portal for other platforms (see Section \ref{sec:DocPortal}).

The LTD branch high-level nodes of the Storage View could be by document type (i.e., guides or documents).
Following the department/owner or specific technote series (e.g., RTN, SITCOMTN) could be the next level of nodes.

\subsubsection{Confluence and Jira}
\label{confluence-jira-storage}

Confluence \citep{Confluence-cite} and Jira \citep{Jira-cite} are part of the Atlassian Corporation suite of tools used by the construction and operations projects for many purposes.
They are collaborative tools where teams/groups can document, share and develop information.
Confluence is organized into \emph{spaces}, each with a varying number of pages and sub-pages.
Jira is organized into \emph{projects}, each of which tracks a list of enumerated tickets or issues.
Both include a large variety of features, tools and extendable add-ons to manipulate or organize the information.
Many of these spaces and projects are very specialized, with some set up for personal use.

On the construction project, there are two instances of each software tool, requiring different credentials --- \url{https://confluence.lsstcorp.org} and \url{https://jira.lsstcorp.org} is for the NSF MREFC effort hosted by LSSTC out of Tucson, Arizona; and, \url{https://confluence.slac.stanford.edu/} and \url{https://jira.slac.stanford.edu/} is for the DOE MIE effort for LSSTCam hosted by SLAC National Accelerator Laboratory out of Stanford, California.
Rubin Observatory operations and pre-operations staff currently use the LSSTC tools.

Currently, the LSSTC Jira instance includes the following normative source of information:
(1) verification elements, plans, cycles, cases, results, etc.,
(2) construction-related risks, opportunities and mitigations,
(3) the Failure Reporting Analysis and Corrective Action System (FRACAS) for failures and corrective actions,
and (4) hazard mitigation verification.

As tools that are heavily used by the project, it is recommended to continue the use of Confluence and Jira into operations.
With a wiki like Confluence which that integrates well with Jira, the two are especially convenient for rapidly developing new ideas, taking and storing meeting minutes, collecting information in interactive tables, and drafting outlines for future documentation, all while tracking the tasks and being able to actively report on status.
Other powerful features include simultaneous editing, native sharing options internal to Confluence/Jira or external such as email, etc.
As operational-based information takes form, departments and technical groups should reconsider the spaces and projects to prevent having a large number of unused or old areas, as seen with the current Confluence and Jira instances.
In the case of Confluence, project staff must be fastidious in moving content needed on a long-term basis into other official storage locations at the earliest appropriate stage (e.g., DocuShare, lsst.io).
Users must understand the limitations of information and provided guidance as to how and when information should be moved from Confluence or Jira to another storage location.

The Confluence branch of the Storage View would begin with the spaces.
Most spaces have a relatively small number of top-level pages which are tailored to a specific need (e.g., separation of subsystems), and these top-level pages can naturally be the next level of nodes.
The Jira branch would begin with the projects, but the next level nodes are not as apparent and should be determined by the owner or responsible group.
There are many ways to create the next level nodes for Jira; for example, most projects have a natural breakdown of structure used to organize relationships between tickets (e.g., epics).
Lower-level nodes will depend on use; they could include items such as meeting notes or customized dashboards.

\subsubsection{Engineering models in Solidworks Product Data Management (PDM)}

Solidworks Product Data Management (PDM) Professional \citep{PDM-cite} is the official computer-aided design (CAD) model repository for the Telescope and Site construction group, including vendor documentation and deliverables.
It uses a check-out / check-in system to allow configuration management of the design where earlier versions can be accessed if needed to compare designs or revert to an earlier design.
A workflow feature allows the designs to go through a review process until the design is approved and locked from further changes.
A revision process is also included in the workflow to allow for changes to the designs if needed after final approval.
The current configuration of the PDM vault contains top-level access to baseline design data and interface control documents (ICDs) (drawings) along with as-designed vendor subsystems.
Solidworks PDM is the normative source of information for system decomposition.

It is recommended to continue use of PDM vault into operations.
In operations, as-built design information will be needed to help with logistics planning for maintenance and future design upgrades.
No information has been deleted or archived at this time.
If continued into operations, it is recommended the vault is reorganized so that legacy data is archived for access but is not easily mistaken for as-built design data.
The specific structure for the branch of the Storage View is highly dependent on this reorganization.
For example, the high-level nodes could be by subsystem or physical location, lifecycle of particular drawings (e.g., as-designed, as-built), or use within specific states of the telescope (e.g., on-sky imaging, preventive maintenance shutdown).

\subsubsection{MagicDraw}

MagicDraw \citep{MagicDraw-cite} is a tool used to maintain a model of the Rubin Observatory system by creating a relational database between system elements.
There are a number of elements captured in the MagicDraw database defined as the normative source of information, with the information synced and/or exported to other repositories.
As the normative source of information, it will be up-to-date when given over from construction and will continue active use within operations.
Details on how Rubin Observatory uses MagicDraw are included in the user guides collected on the following Confluence page: \url{https://confluence.lsstcorp.org/display/SYSENG/MagicDraw+LSST+Users+Guide}.

\begin{itemize}
	\item \textbf{Hazard Analysis} --- normative source --- synced to Jira for verification tasks.
	\item \textbf{Failure Mode and Effects Analysis (FMEA)} --- normative source --- known to be incomplete.
	\item \textbf{Requirements} --- normative source --- exports to DocuShare.
	\item \textbf{Verification Elements} --- normative source --- synced to/from Jira.
	\item \textbf{Verification Plans/Cycles/Cases} --- synced to/from Jira via Syndeia\texttrademark\ \citep{syndeia-cite}.
	See Section \ref{confluence-jira-storage} for normative source).
	\item \textbf{SAL Commands, Events, Telemetry} --- imported from CSC XML.
	\item \textbf{Operations Concepts} --- source of truth.
	\item \textbf{System-level State Machine} --- source of truth.
	\item \textbf{Interlocks} --- modeled from source material.
	\item \textbf{Structural Decomposition} --- will be synced from Solidworks in the future.
\end{itemize}

It is recommended to continue the use of MagicDraw into operations.
The user guides should be relocated into a more appropriate storage location, such as lsst.io.
The content in MagicDraw should be reviewed to create the tree for the Storage View that can readily reflect normative sources of information and how/what references or depends on this information.
It is crucial that the normative source of information is clearly defined between MagicDraw and other repositories which are synced or used for archival purposes, as well as the workflow and method to do so.
The effort required to complete the FMEA information will depend on the level of completeness when handed over to operations.
Effort to sync the structural decomposition with Solidworks can be evaluated when the system information is more complete.

\subsubsection{Euporie}

Euporie is a network drive on a server managed by the construction project; and, it is accessible with Rubin Observatory credentials through VPN at \url{smb://euporie}.
The drive contains a number of personal directories and a directory named "TS-Deliverables" managed by the Telescope and Site group.
Stored in subdirectories of TS-Deliverables are vendor-supplied documentation as contract deliverables (design documentation, construction drawings, manuals and other miscellaneous information), with each subdirectory.

It is recommended to discontinue use of Euporie for operations.
A review of the content of personal and shared directories is recommended to determine what information should be relocated on a case-by-case basis.
Additionally, a determination should be made if information should be archived as construction-related or moved to a more active storage location for operations staff access.
After this is complete, and if access will continue for purpose unrelated to operations, users must understand the limitations of information (namely for long-term operational activities) and provided guidance as to how and when information should be moved from Euporie to another storage location.

\subsubsection{GitHub}

GitHub\texttrademark\ \citep{GitHub-cite} is used by the project for software and documentation collaboration, storage, version control via git\texttrademark\ \citep{git-cite} and deployment.
GitHub is primarily utilized via \emph{repositories}, or repos.
Repositories are owned by individual users or an \emph{organization}, and organizations can include \emph{teams} for additional granularity.

The Rubin Observatory construction and operations projects use a large number of GitHub repositories, organizations and teams, primarily to ease access control.
The main project GitHub organization is \url{https://github.com/lsst}; however, not all operations software is found in this organization.
GitHub organizations with operational software and documentation will likely transition smoothly to Operations teams.
Particular attention will need to be paid to maintaining organizations that no longer have an active team associated with them.

As a heavily used tool by the project, it is recommended to continue the use of GitHub via git into operations.
Effort by construction, pre-operations and operations teams already implemented infrastructure for utilization and the GitHub collaboration structure, including basic structure for operations.
There are a large number of repositories containing the normative source of information, and the project should ensure these repositories are clearly indicated as such and what respective information is the normative source.

\subsubsection{Drupal}

Drupal \citep{drupal-cite} is an open source web content management framework used for project websites’ back-end, including \url{https://www.lsst.org}, \url{https://project.lsst.org} and sites created to facilitate project meetings and reviews.
To avoid the complication of having to make multiple site updates when content changes, the documents on each of these sites are served by hyperlinks that pull files from whatever repositories contain their normative sources of information.
While it is possible to upload discrete files to a specific site’s server location, by policy and standard the project eschews doing so.

The major exception is project-level meetings and reviews sites, such as the Project and Community Workshop (PCW) or agency reviews.
Their site directories contain document and presentation files in order to preserve the content presented at the time.
At the conclusion of the event, these files are uploaded to DocuShare in a collection specific to the event.
Documents such as policies, requirements, and design documents are uploaded as a ZIP file to preserve a snap-shot while preventing replication and multiple handles that may cause confusion with the normative source of information.

It is recommended for the project to evaluate the need for web content management in operations.
If needed, the project should consider if other tools planned for operations would fulfill the same need, or if Drupal should be continued into operations.
The level of change control appropriate for websites must be determined, as well as files in site directories or hyperlinked files without change control.
Since the method described for project-level meetings and reviews may not be implemented by all subsystems on the construction project, it is recommended the project review the content to ensure historical information is preserved appropriately.
If Drupal is discontinued, it is recommended the project review if all required archival information is transferred to the new web content management system and/or an actively supported repository (e.g., DocuShare).
If web content management is controlled to the same degree as construction, it is recommended that the project provide a clear workflow of how and where normative sources of information is used and stored.

\subsubsection{LSSTCam information}

During construction, the Camera used a set of web-based tools used across multiple sites during the testing and construction of the Camera and its subcomponents --- eTraveler, Data Portal and Data Catalog.
Databases and servers managed by SLAC used these tools to capture procedure, testing data, track component/assembly status and information, and to track the acceptance and non-conformance reports.
Both the "prod" and "dev" instances include information on production parts.
Additionally, there is construction information stored in a SLAC-hosted instance of Confluence and Jira.

It is recommended the project review the information to determine the type of information and method by which it is archived to an actively supported repository (e.g., DocuShare).
Data within repositories that are not be actively supported or available may be lost.
For repositories that may not be available throughout operations, the project should consider which information should be transferred for future use in operations or archived for future retrieval.

\subsubsection{Primavera P6}

Primavera P6\textregistered\ is a project portfolio management tool by Oracle\textregistered\ Corporation. \citep{P6-cite}
It was used for planning, managing, and executing the Construction project.
It is recommended that the P6 record of the Construction project be exported and saved at the end of the Project; its use will not continue into Operations.

\subsubsection{Verification reports}

The Rubin Observatory Verification Architecture used for construction and commissioning activities is composed of our MagicDraw model, Jira with Test Manager application, and Syndeia to synchronize data between MagicDraw and Jira.
For all verification events performed by the SIT-COM team, a SIT-COM Test Plan/Report (SCTR) is created setting the standard for verification execution and results.
The SCTR/DMTR is generated using the Jira REST API pulling the information from the verification objects.
Formal detailed records of each verification done by Rubin Observatory is in the LSST Verification and Validation Jira project, with reports published to lsst.io as a SCTR or DMTR in addition to archiving it into DocuShare.

Similar to MagicDraw, it is crucial that the normative source of information is clearly defined between the verification reports and other repositories which are synced or used for archival purposes, as well as the workflow and method to do so.

\subsubsection{Computerized Maintenance Management System, CMMS}

The Rubin Construction project selected openMAINT\textregistered\ as the customizable Computerized Maintenance Management System (CMMS), \citep{openmaint-cite},  for use throughout the entirety of the LSST.  
The managing company, Tecnoteca srl\copyright will provide consultancy and initial setup support. 

The CMMS is expected to execute and coordinate maintenance activities, maintain the history of these activities, and provide reporting tools.
It is expected that not all procedures will be stored in CMMS, e.g., a hyperlink to a procedure in DocuShare may be used instead of writing the procedure into CMMS.
As the work release workflow will be in Jira, the CMMS will have an interface (e.g., REST API) and update Jira as needed.
With a well-defined workflow, the CMMS and Jira interactions can be streamlined to minimize confusion of scope and use.

It is recommended that the project review the expected capabilities of the CMMS well before the start of operations, possibly as part of one of the early construction closeout reviews, so it can be determined what type of information will be in CMMS as the normative source of information.

It is crucial that the normative source of information is clearly defined between the CMMS and other repositories, e.g. FRACAS, which are synced or used for archival purposes, as well as the workflow and method to do so.

\subsubsection{NOIRLab Alcea Risk Tool}

The Rubin Observatory is required to include risks, opportunities and mitigations in the NOIRLab managed risk management software tool, Alcea Tracking Solutions software \url{https://noirlab.alceatech.com/}.
The Rubin Observatory information is exported and archived in GitHub by operations staff, and a user guide for the Rubin Observatory is available at \href{https://rtn-051.lsst.io/}.
Access will be limited to those using the tool directly and designees.
Given the limited access, the project expects to use Jira for detailed planning and analysis.

It is recommended the project should confirm that the normative source of information for risk, opportunities and mitigations will be Jira.
Assuming Jira will be used, the type or criteria of risks, opportunities and mitigations required to be included in the Alcea Risk Tool must be defined and agreed upon between the project and NOIRLab.
Additionally, it is recommended that a workflow, method and/or software is used to ensure information is consistently synced at an agreed upon frequency.

