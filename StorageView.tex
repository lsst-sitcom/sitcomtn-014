\subsection{Storage View}
\label{sec:storage-view}

The purpose of the \emph{Storage View} is to capture the official repositories retaining, archiving, organizing, and accessing Rubin Observatory technical documentation in a reliable, consistent manner for users, developers and management.
As a fundamental corollary, the Storage View provides a method for departments and technical groups to define the normative source for information, i.e., a single, definitive source of truth.
With canonical information storage, others can develop associations, links and references to impart information to inter-departmental or external documents or groups, allowing a reduction in replication, reliable information flow, and preservation of informational dependencies.
Unless mandated by the project, the department or responsible group can choose the manner and method of storing information and transposing information to other locations, in conjunction with the stakeholder(s).
The primary users are managers, engineers, specialists, technicians, web development staff, system administrators, and documentation specialist.

To ensure a high degree of reliability, it is crucial staff store all operational and historical information within officially designated locations.
Specifying official repositories will limit the level of effort in the utilization and maintenance of multiple documentation systems, prevent or highly discourage using non-official storage repositories, and reduce risk to storing and accessing operational and historical information.
By limiting the available repositories, effort and resources can focus on best using the official storage locations or modify them in an orderly, consistent manner.

As a shared responsibility between system administrators and departments, all official repositories should be well maintained.
System administrators must ensure access throughout operations (e.g., readily available and backed-up) and a level of expertise must be on-hand for maintenance, issues and general assistance to staff.
Departments and staff must maintain their information and its organization within the repositories.
They should recognize and act when information or utilization becomes outdated or unused as to prevent large amounts of depreciated content, stress on storage and bit rot.
This is especially important to the development, deployment and maintenance of the Documentation Portal architecture (Section \ref{sec:DocPortal}), as leveraging specific use cases and repository's native or project-developed metadata fields are key components.
Additionally, this effort will better suit the project and departments to formalizing a long-term organizational structure of each repository, create customized workflows, transition to the Documentation Views, and implement new or updated documentation.

Non-official storage locations pose unique operational and managerial risks which can lead to information and data loss, increased information security risks, and impact to the project's schedule and budget.
This includes information control (e.g., access, enforced version control, archiving), reliable data recovery, unknown or unplanned risks, and lack of available support (e.g., available labor, expertise).
Platforms may become unsupported or unavailable and the information therein becomes lost or otherwise inaccessible.
This risk has already been realized within the construction project, although of minor impact, as evident from a small number databases (e.g., personal drives, old network drives) that are difficult to access or have lost or inaccessible information. \citedsp{SITCOMTN-012}
Additionally compounding the risk to information access, it becomes increasingly difficult to share information, especially when the repository is ill-defined, uncertainty of what replicated information is the most current, and uncertainty of which repository is the normative source of information.
Without a discrete set of storage locations, it is impossible to guarantee reliable and navigable information or data across multiple platforms throughout operations.

The Storage View tree is similar to a product tree, with the basis of design a result of the \citeds[Construction Documentation Inventory]{SITCOMTN-012}.
This section summarizes some important details from the document, but \citeds{SITCOMTN-012} includes more information such as the inventory for each repository.
Each repository is a root-node, similar to a Level-1 product, and the first set of leaf-nodes captures the location's organizational structure and its metadata structure.
Critically, the Storage View trees should readily indicate what are the normative sources of information so other Documentation Views can reliably refer to the accurate and up-to-date information.
It will be the responsibility of the department or technical group to define the organizational structure, with documentation specialists assisting in developing a consistent approach across the project and sharing lessons learned between groups.
Software developers require the metadata and its structure to develop the Documentation Portal, including the development of additional metadata fields with assistance from system administrators.
Metadata information was not surveyed by the Documentation Working Group; however, metadata fields and structure must be documented for development, changes and implementation of the Documentation Portal.
Recommendations and suggestions by the Documentation Working Group to transition and use the respiratory in operations are provided, including some suggested repository's organizational structure.
The level of effort and schedule impacts are included where appropriate.

\subsubsection{DocuShare}

Xerox\textregistered\ DocuShare\textregistered\ content management system \citep{DocuShare-cite} is the Construction Project's official document repository.
It was selected during the design and development phase to meet the NSF requirement for a document management system.
The Construction Project Office expects DocuShare to be the repository for official versions of management policies, plans and procedures, design documents, safety documentation, hazard analyses, released requirements and interface control documents generated from the SysML model, and project standards, guidelines and templates (this list is not intended to be exhaustive).

Three of DocuShare's advantages are handles, version control and co-location.
Each object has a unique identifier called a handle, which follows the object regardless of versioning or location(s) in the directory structure (called collections).
Each handle has a version history that lists all previously uploaded files for the object in question.
One of those versions is designated as the preferred version, which represents the document's official, approved version and is served by the database when clicking the object's title or from a properly formatted URL shared or hosted outside of DocuShare.
The object's handle does not change when/if a new version of the document is created.
Lastly, objects can appear simultaneously in as many collections as are necessary and/or relevant to the document.
This is facilitated by an object's "locations" property; locations are added as appropriate, and the handle automatically appears in the newly added collection.
The combination of handles, versioning and co-location creates a system where each document is represented by a single record, avoiding duplication and/or version confusion.

The web interface for DocuShare is \href{https://docushare.lsstcorp.org/docushare/dsweb/HomePage}.
Currently, DocuShare contains more than $30,000$ documents in more than $10,000$ collections.
Creation, retention and version control of those documentation classes generally have been well managed; however, the bulk of DocuShare documents likely represents objects created ad hoc by general project staff for specific purposes.
In addition, there are thousands of pieces of work product that may have archival significance but likely will not be useful for Operations.

As a tool heavily used by the project, it is recommended to continue the use of DocuShare into operations.
Operations should use the DocuShare instance currently in use by the construction project with the Archive Server add-on enabled, as the continuity afforded by doing so is important and useful.
Under this model, the construction project content would be moved to the archive server, and the active server would contain only operations and operations-relevant construction content following an agreed-upon directory structure.
Until this transition is completed project-wide, the current location for operational documentation is \href{https://docushare.lsstcorp.org/docushare/dsweb/View/Collection-602}{Collection-602}.
Documentation specialists have been encouraging sub-system staff to organize objects in this collection and store operations information therein.
In the few cases of feedback, this new setting has been positively received, citing the DocuShare structure for construction is difficult to understand and unintuitive for new staff, especially for operations staff.

The suggested DocuShare branch of the Storage View for operations starts with the a collection for each department.
Additional collections can be created based on the project's needs as a whole (e.g., interdepartmental items such as widely used software tools).
The next level of nodes is defined by the system's and/or product's department or owner.
Further lower-level nodes would depend on how the department and owners want to parse the information, e.g., separation by subsystem, separation by topics such as reports, or a combination of the two.
As for construction documentation, it is recommended to have a consistent organizational structure when transferring over to the archive server.
The archiving process should include a determination by the department if the information is useful in operations (determining if it should be archived) and if the information should be queryable in the archive (e.g., raw data).
This is especially topical for vendor documentation and deliverables.

See Section \ref{sec:docushare-path} for a detailed discussion on the reasoning, recommended actions and expected resources to implement and maintain DocuShare in operations.
A more detailed analysis of arguments and considerations for this proposal's recommendation is available in DocuShare Options Trade Study for the Documentation Working Group. \citedsp{Document-36788}

\subsubsection{LSST the Docs (www.lsst.io)}

LSST the Docs (LTD), also known by its URL "lsst.io", is a documentation hosting platform built and operated by the SQuaRE team within the Data Management group.
LTD hosts \emph{versioned static websites}, meaning any website built from HTML, CSS and JavaScript that doesn't need an active server to render content (as opposed to say Confluence, DocuShare, or Drupal websites).
Static websites are a natural fit for documentation projects that originate from repositories hosted by \href{https://www.github.com}{GitHub} \citep{GitHub-cite}, so LTD is uniquely developed to be built around versioned documentation in GitHub.
\emph{lsst.io} is one such deployment used as a hosting domain for Rubin Observatory, where all subdomains of lsst.io are independent documentation projects.
The technical motivation and design of LTD are documented in \href{https://sqr-006.lsst.io}{SQR-006: The LSST the Docs Platform for Continuous Documentation Delivery}. \citeds{SQR-006}
The key technical features of LTD are:

\begin{itemize}
	\item high reliability, scaling, and security: documentation is hosted in Amazon S3 and served through the Fastly content distribution network.
	We don't operate any servers that receive traffic from users;
	\item versioned documentation; and
	\item flexibility to host any type of static website.
\end{itemize}

Using LTD documents provides a simple use case with additional features developed by the SQuaRE team.
The root URL for a documentation project hosts the "default" version, which has a configurable meaning for each project (such as software release versions, temporary collaborative drafts, or an active version in DocuShare).
Users can also browse other versions of the documentation through the "/v/" dashboard pages (for example \url{https://www.lsst.io/v/}).

LTD hosts two types of documentation projects: \emph{guides} and \emph{documents}.
Guides are multi-page websites convenient for user interaction and navigation (e.g., \href{developer.lsst.io}{Data Management Developer Guide}, T\&S software guides at \url{https://obs-controls.lsst.io}).
Documents are "single-page" artifacts, analogous to documents that might be found in DocuShare, and they are sometimes referred to as \emph{technical notes} (shortened to "technote" or an appended "TN") --- see \href{https://sqr-000.lsst.io}{SQR-000: The LSST DM Technical Note Publishing Platform} for the motivation to create technical notes. \citedsp{SQR-000}
Though not required, guides are generally authored using an open-source tools using themes that is maintained by the SQuaRE team via documenteer \cite{documenteer-cite} (see \url{https://documenteer.lsst.io/} for corresponding guide).
Besides documentation tied to specific software projects or services, guides can also collect procedures for teams, see the DM Developer Guide (\url{https://developer.lsst.io}) or the Observatory Operations Documentation (\url{https://obs-ops.lsst.io}).

DM has embraced the use of LTD by hosting a guide on lsst.io for every software project or service.
Further, DM has developed most of its change-controlled documents (LDM) on its lsst.io site to take advantage of the sophisticated collaboration features that GitHub offers (for an example, see \url{https://ldm-151.lsst.io}).
Change-controlled documents are submitted to DocuShare for archival once approved using a release process mediated through GitHub, Jira, and the relevant control board.
LTD is currently hosting documents from the DMTR, LDM, LPM, LSE, and SCTR document series (note that this includes test and verification reports).

A prime example of the LTD and GitHub tools used by the project for operations activities is the T\&S Commandable Service Abstraction Layer (SAL) Component (CSC) XML package and user guides, as summarized by \url{https://ts-xml.lsst.io}.
The CSC data objects are defined in XML for SAL to consume and produce language-specific libraries that enable communication over the Data Distribution System (DDS) network.
These XML files are critical for defining the configuration and interaction of the systems within the Summit Facility observing environment.

A unique example is the homepage for the LTD documentation platform, \url{https://www.lsst.io}, which serves as a portal for LTD indexed documentation for searches and faceted browsing capabilities. \citep{lsst.io-cite}
Users can search across metadata and full text (this feature is powered by the commercial service Algolia \citep{Algolia-cite} in conjunction with a scraper bot built by SQuaRE) or browse through curated collections.
The site itself is built with React/Gatsby.js (\url{https://github.com/lsst-sqre/www_lsst_io}), the search database is SaaS \citep{SaaS-cite}), and the bot that indexes content into the search database is called Ook (\url{https://github.com/lsst-sqre/ook}).
It is still in development and the current status is documented at \url{https://www.lsst.io/about/}.

As a customized set of tools that is heavily used by the project, it is recommended to continue the use of LTD and its software system into operations.
The project should use documents and technotes to capture information that is static, rarely changed or serving a temporary need (e.g., proposals, documenting proof or principles, status updates);
whereas, guides should be used for information that is actively updated with current information (e.g., troubleshooting guides, procedures not under change control).
Neither LTD documentation type should be used for documents under change control albeit one can link to the change-controlled document within a document or guide.
Further, as the LTD software system, it's additional utilities and \url{https://www.lsst.io} are designed and built by the SQuaRE team, there is considerable opportunity to leverage the expertise and experience already developed to implement a documentation portal for other platforms (see Section \ref{sec:DocPortal}).

The LTD branch high-level nodes of the Storage View could be by document type (i.e., guides or documents).
Following the department/owner or specific technote series (e.g., RTN, SITCOMTN) could be the next level of nodes.

\subsubsection{Confluence and Jira}
\label{confluence-jira-storage}

Confluence \citep{Confluence-cite} and Jira \citep{Jira-cite} are part of the Atlassian Corperation suite of tools used by the construction and operations projects for many purposes.
They are collaborative tools where teams/groups can document, share and develop information.
Confluence is organized into \emph{spaces}, each with a varying number of pages and sub-pages.
Jira is organized into \emph{projects}, each of which tracks a list of enumerated tickets or issues.
Both include a large variety of features, tools and extendable add-ons to manipulate or organize the information.
Many of these spaces and projects are very specialized, with some set up for personal use.

On the construction project, there are two instances of each software tool, requiring different credentials --- \url{https://confluence.lsstcorp.org} and \url{https://jira.lsstcorp.org} is for the NSF MREFC effort hosted by LSSTC out of Tucson, Arizona; and, \url{https://confluence.slac.stanford.edu/} and \url{https://jira.slac.stanford.edu/} is for the DOE MIE effort for LSSTCam hosted by SLAC National Accelerator Laboratory out of Stanford, California.
Rubin Observatory operations and pre-operations staff currently use the LSSTC tools.

Currently, the LSSTC Jira instance includes the following normative source of information:
(1) verification elements, plans, cycles, cases, results, etc.,
(2) construction-related risks, opportunities and mitigations,
(3) the Failure Reporting Analysis and Corrective Action System (FRACAS) for failures and corrective actions,
and (4) hazard mitigation verification.
However, it is required operations include risks, opportunities and mitigations in the NOIRLab managed risk management software tool, Alcea Tracking Solutions software \url{https://noirlab.alceatech.com/saml2/sso}.
The project must determine if the normative source for information on risk, opportunities and mitigations will continue to be Jira or moved to the NOIRLab risk management tool.
If Jira is the normative source of information for these items, the project must develop a method to sync information to the NOIRLab tool.

As tools that are heavily used by the project, it is recommended to continue the use of Confluence and Jira into operations.
With a wiki like Confluence which that integrates well with Jira, the two are especially convenient for rapidly developing new ideas, taking and storing meeting minutes, collecting information in interactive tables, and drafting outlines for future documentation, all while tracking the tasks and being able to actively report on status.
Other powerful features include simultaneous editing, native sharing options internal to Confluence/Jira or external such as email, etc.
As operational-based information takes form, departments and technical groups should consider the spaces and projects to prevent having a large number of unused or old areas, as seen with the current Confluence and Jira instances.
In the case of Confluence, project staff must be fastidious in moving content needed on a long-term basis into other official storage locations at the earliest appropriate stage (e.g., DocuShare, lsst.io).
Users must understand the limitations of information and provided guidance as to how and when information should be moved from Confluence or Jira to another storage location.

The Confluence branch of the Storage View would begin with the spaces.
Most spaces have a relatively small number of top-level pages which are tailored to a specific need (e.g., separation of subsystems), and these top-level pages can naturally be the next level of nodes.
The Jira branch would begin with the projects, but the next level nodes are not as apparent and should be determined by the owner or responsible group.
There are many ways to create the next level nodes for Jira; for example, most projects have a natural breakdown of structure used to organize relationships between tickets (e.g., epics).
Lower-level nodes will depend on use; they could include items such as meeting notes or customized dashboards.
The NOIRLab risk tool should also have a branch in the Storage View.

\subsubsection{Engineering models in Solidworks Product Data Management (PDM)}

Solidworks Product Data Management (PDM) Professional \citep{PDM-cite} is the official computer-aided design (CAD) model repository for the Telescope and Site construction group, including vendor documentation and deliverables.
It uses a check-out / check-in system to allow configuration management of the design.
Each check-in produces a new version of the part or assembly.
Earlier versions can be accessed if needed to compare designs or revert to an earlier design.
A workflow feature allows the designs to go through a review process until the design is approved and locked from further changes.
A revision process is also included in the workflow to allow for changes to the designs if needed after final approval.
The software allows for vault replication at multiple sites and the project currently has a server operating in Tucson, Arizona and Chile to support CAD users at multiple sites.
Solidworks PDM is the normative source of information for system decomposition.

The current configuration of the PDM vault contains top-level access to baseline design data and interface control documents (ICDs) (drawings) along with as-designed vendor subsystems.
This includes the original baseline design, early designs and further development throughout construction.
In addition, a series of design and drafting standards is also stored in the PDM system.
Two servers are hosted: the main server located in Tucson, Arizona, and a replicated server in La Serena, Chile to allow faster file access.

In operations, as-built design information will be needed to help with logistics planning for maintenance and future design upgrades.
No information has been deleted or archived at this time.
The Documentation Working Group recommends the vault is reorganize so that legacy data is archived for access but is not easily mistaken for as-built design data.
This will require a full-time resource for 3 to 6 months.
The specific structure for the branch of the Storage View is highly dependent on this reorganization.
For example, the high-level nodes could be by subsystem or physical location, lifecycle of particular drawings (e.g., as-designed, as-built), or use within specific states of the telescope (e.g., on-sky imaging, preventive maintenance shutdown).

See Section \ref{sec:pdm-path} for a detailed discussion on the reasoning, recommended actions and expected resources to implement and maintain Solidworks PDM in operations.

\subsubsection{MagicDraw}

MagicDraw \citep{MagicDraw-cite} is a tool used to maintain a model of the Rubin Observatory system by creating a relational database between system elements.
There are a number of elements capture in the MagicDraw database defined as the normative source of information, with the information synced and/or exported to other repositories.
As the normative source of information, it will be up-to-date when given over from construction and continue active use within operations.
Details on how Rubin Observatory uses MagicDraw are included in the user guides collected on the following Confluence page: \url{https://confluence.lsstcorp.org/display/SYSENG/MagicDraw+LSST+Users+Guide}.

\begin{itemize}
	\item \textbf{Hazard Analysis} --- normative source --- synced to Jira for verification tasks.
	\item \textbf{Failure Mode and Effects Analysis (FMEA)} --- normative source --- known to be incomplete.
	\item \textbf{Requirements} --- normative source --- exports to DocuShare.
	\item \textbf{Verification Elements} --- normative source --- synced to/from Jira.
	\item \textbf{Verification Plans/Cycles/Cases} --- synced to/from Jira via Syndeia\texttrademark\ \citep{syndeia-cite}.
	See Section \ref{confluence-jira-storage} for normative source).
	\item \textbf{SAL Commands, Events, Telemetry} --- imported from CSC XML.
	\item \textbf{Operations Concepts} --- source of truth.
	\item \textbf{System-level State Machine} --- source of truth.
	\item \textbf{Interlocks} --- modeled from source material.
	\item \textbf{Structural Decomposition} --- will be synced from Solidworks in the future.
\end{itemize}

It is recommended to continue the use of MagicDraw into operations.
The user guides should be relocated into a more appropriate storage location, such as lsst.io.
The content in MagicDraw should be reviewed to create the tree for the Storage View that can readily reflect normative sources of information and how/what references or depends on this information.
The effort required to complete the FMEA information will depend on the level of completeness when handed over to operations.
Effort to sync the structural decomposition with Solidworks will be evaluated when the system information is more complete.

\subsubsection{Euporie}

Euporie is a network drive on a server managed by the construction project; and, it is accessible with Rubin Observatory credentials through VPN at \url{smb://euporie}.
The drive contains a number of personal directories and a directory named "TS-Deliverables" managed by the Telescope and Site group.
Stored in subdirectories of TS-Deliverables are vendor-supplied documentation as contract deliverables (design documentation, construction drawings, manuals and other miscellaneous information), with each subdirectory.

It is recommended to discontinue use of Euporie for operations.
A review of the content of personal and shared directories is recommended to determine what information should be relocated on a case-by-case basis.
Additionally, a determination should be made if information should be archived as construction-related or moved to a more active storage location for operations staff access.
After this is complete and access will continue for purpose unrelated to operations, users must understand the limitations of information (namely for long-term operational activities) and provided guidance as to how and when information should be moved from Euporie to another storage location.

\subsubsection{GitHub}

GitHub\texttrademark\ \citep{GitHub-cite} is used by the project for software and documentation collaboration, storage, version control via git\texttrademark\ \citep{git-cite} and deployment.
GitHub is primarily utilized via \emph{repositories}, or repos.
Repositories are owned by individual users or an \emph{organization}, and organizations can include \emph{teams} for additional granularity.

The Rubin Observatory construction and operations projects use a large number of GitHub repositories, organizations and teams, primarily to ease access control.
The main project GitHub organization is \url{https://github.com/lsst}; however, not all operations software is found in this organization.
On the construction project, each subsystem on the construction project generally has its own GitHub organization, though in some cases subsystems have additional GitHub organizations for specific teams or projects.
For example, the Science Platform software is contained in \url{https://github.com/lsst-sqre}, Telescope \& Site software in \url{https://github.com/lsst-ts}, EPO software in \url{https://github.com/lsst-epo} and Camera software in several organizations.

As a heavily used tool by the project, it is recommended to continue the use of GitHub via git into operations.
Effort by construction, pre-operations and operations teams already implemented infrastructure for utilization and the GitHub collaboration structure, including basic structure for operations.
There are a large number of repositories containing the normative source of information, and the project should ensure these repositories are clearly indicated as such and what respective information is the normative source.

\subsubsection{Drupal}

Drupal \citep{drupal-cite} is an open source web content management framework used by the project for its websites’ back-end, including \url{https://www.lsst.org}, \url{https://project.lsst.org} and sites created to facilitate project meetings and reviews.
To avoid the complication of having to make multiple site updates when content changes, these documents of each of these sites are served by hyperlinks that pull files from whatever repositories contain their normative sources of information.
While it is possible to upload discrete files to a specific site’s server location, by policy and standard the project eschews doing so.

The major exception is regarding project-level meetings and reviews sites, such as the Project and Community Workshop (PCW) or agency reviews.
Their site directories contain document and presentation files in order to preserve the content presented at the time.
At the conclusion of the event, these files are uploaded to DocuShare in a collection specific to the event.
Documents such as policies, requirements, and design documents are uploaded as a ZIP file to preserve a snap-shot while preventing replication and multiple handles that may cause confusion with the normative source of information.
The method described for project-level meetings and reviews may not be implemented by all subsystems on the construction project, and the project must consider if and how to archive this information.

The project must consider if and how its appropriate to continue using Drupal for web content management into operations.
If used in operations, it must be clear how and where normative sources of information are used (e.g., DocuShare).
It must be determined the level of change control appropriate for websites, as well as files in site directories or hyperlinked files without change control.

\subsubsection{LSSTCam information}

There are multiple storage locations for information, documents and data produced during the MIE effort for LSSTCam:



\subsubsection{Engineering Facility Database (EFD)}



\subsubsection{Primavera P6}



\subsubsection{Verification reports}



\subsubsection{CMMS}



\subsubsection{NOIRLab Risk Tool}


