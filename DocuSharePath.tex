\subsection{Path Forward with DocuShare}
\label{sec:docushare-path}

As part of the proposal, the Documentation Working Group considered and evaluated options for Xerox\textregistered\ DocuShare\textregistered\ \citep{DocuShare-cite} use within operations.
It is expected it will continued to be used in operations.
It is recommended that Rubin Observatory Operations should use the DocuShare instance currently in use by the construction project with the Archive Server add-on enabled, as the continuity afforded by doing so is important and useful.
DocuShare’s version control, version history, permissions and co-location capabilities are valuable tools for effective document management and should be maintained.
Additionally, continuing DocuShare use will ease the transition from construction to operations by providing a single shared platform, thereby avoiding compatibility issues, eliminating the need to migrate documents, preventing loss of documentation and avoiding confusion by retaining legacy handles and references.
Enabling the Archive Server will allow operations staff to structure and manage DocuShare according to its needs while retaining access to all documentation accumulated during construction.
The Documentation Working Group believes this recommendation’s benefits outweighs any required new and/or continuing costs and labor resources for setup and ongoing maintenance.

The recommendation is consistent with NOIRLab document management, which has selected DocuShare as the lab’s document repository.
By also using DocuShare, the repository for Rubin Observatory Operations will be compatible with the knowledge and skills held by the business unit responsible for NOIRLab document management.
While it's possible to use NOIRLab’s DocuShare, the Documentation Working Group believes that to be less advantageous than retaining the construction project DocuShare instance:
switching to NOIRLab’s DocuShare would result in loss of legacy document handles and would require labor-intensive and time-consuming document migration.
Continuing to use the current DocuShare will retain document version histories and allow persistence of document handles for foundation documents, many of which are known within the project and community as much by their handles as they are by their titles.
Lastly, each instance of DocuShare has a $2,000,000$ document limit.
If the construction project DocuShare instance is retained, the entire document limit is available to Rubin Observatory.
If NOIRLab’s DocuShare instance is used, Rubin would have to share the document limit with other lab units and programs.

However, the Documentation Working Group recognizes that simply continuing with the current DocuShare instance as-is fails to address long-standing issues of clutter, inadequate metadata, and poor classification;
therefore, we recommend enabling the Archive Server add-on.
The add-on provides a second server into which documents whose lifecycles have ended can be sequestered, freeing the active server to be structured and managed according to best practices and needs for operational activities.
While archived documents no longer reside in the active server and are not represented in its directory structure, they are viewable and searchable and can be restored to the active server if necessary.
The two servers are connected and can be interacted with using a single user interface.
Under this model, the construction project content would be moved to the archive server, and the active server would contain only operations and operations-relevant construction content following an agreed-upon directory structure.

Further, to mitigate the risk of replicating the clutter and less-than-optimal classification of construction-era DocuShare, the working group recommends Rubin Operations adopt a more formal workflow for how and by whom DocuShare content is created, formatted, and updated.
This workflow would involve something similar to NOIRLab’s steward concept.
NOIRLab’s document management plan expects each business unit or program to designate a person or persons to lead the group’s documentation activities in DocuShare.
These stewards are to be highly trained so they can perform DocuShare functions, provide guidance to others, enforce standards, and correct errors.
The idea is to have fewer but better-trained users with full DocuShare permissions and functionality.
While this will result in a less distributed workload, consistency and quality will improve, leading to greater tool utility.
The Operations departments should follow a similar approach, and as much as possible, the workflow should occur within DocuShare.

Retaining the current DocuShare instance and enabling the Archive Server add-on requires both new and continuing costs and labor.
The calendar year 2020 renewal cost for the current DocuShare instance was \$22,000, and Rubin's IT resources are expended to maintain the service, which is hosted on a physical server in Tucson, Arizona.
The archive server add-on carries a one-time cost of \$7,500 and an additional \$1,400 per year for support, according to a December 
2020 quote from Xerox\textregistered\ .
Additionally, IT resources will be needed to configure the add-on, and some training likely will be required.
Despite these considerations, the Documentation Working Group believes the benefits outweigh the costs, and the required resources are less burdensome than those that would be needed to adopt another repository.

Further information on current DocuShare use is detailed in \citedsp{SITCOMTN-012}.
A more detailed analysis of arguments and considerations for this proposal's recommendation is available in DocuShare Options Trade Study for the Documentation Working Group. \citedsp{Document-36788}
