\subsection{The Path Forward with DocuShare}

The working group recommends that Rubin continue to use DocuShare in Operations; further, we recommend that Operations use the DocuShare instance
currently in use by Construction with the Archive Server add-on enabled. The working group believes the continuity afforded by doing so is important and useful.
DocuShare’s version control, version history, permissions and co-location capabilities are valuable tools for effective document management and should be 
maintained. Additionally, continuing DocuShare use will ease the transition from Construction to Operations by providing a single shared platform, thereby 
avoiding compatibility issues, eliminating the need to migrate documents, preventing loss of documentation and avoiding confusion by retaining legacy handles 
and references. Enabling the Archive Server will allow Operations to structure and manage DocuShare according to its needs while retaining access to all 
documentation accumulated during Construction. The working group believes the recommendation’s benefits outweigh any required new and/or continuing costs 
and labor resources.

The recommendation is consistent with NOIRLab document management, which has selected DocuShare as the lab’s document repository. By also using 
DocuShare, Rubin Operations’ repository will be compatible with the knowledge and skills held by the business unit responsible for NOIRLab document 
management. While Rubin Operations has the option to use NOIRLab’s DocuShare, the working group believes that to be less advantageous than retaining 
Rubin Construction’s DocuShare instance. Switching to NOIRLab’s DocuShare would result in loss of legacy document handles and would require labor-
intensive and time-consuming document migration. Continuing to use Rubin DocuShare will retain documents’ version histories and allow persistence of handles 
for foundation documents, many of which are known within the project and community as much by their handles as they are by their titles. Lastly, each instance 
of DocuShare has a $2,000,000$ document limit. If the Construction project DocuShare instance is retained, the entire document limit is available to Rubin. If 
NOIRLab’s DocuShare instance is used, Rubin would have to share the document limit with other lab units and programs.

However, the working group recognizes that simply continuing with the Rubin DocuShare as-is fails to address long-standing issues of clutter, inadequate 
metadata, and poor classification; therefore, we recommend enabling the Archive Server add-on. The add-on provides a second server into which documents 
whose lifecycles have ended can be sequestered, freeing the active server to be structured and managed according to best practices and Operations’ needs. 
While archived documents no longer reside in the active server and are not represented in its directory structure, they are viewable and searchable and can be 
restored to the active server if necessary. The two servers are connected and can be interacted with using a single user interface. Under this model, 
Construction project content would be moved to the archive server, and the active server would contain only Operations and Operations-relevant Construction 
content following an agreed-upon directory structure.

Further, to mitigate the risk of replicating the clutter and less-than-optimal classification of Construction-era DocuShare, the working group recommends Rubin 
adopt a more formal workflow for how and by whom DocuShare content is created, formatted, and updated. This workflow would involve something similar to 
NOIRLab’s steward concept. NOIRLab’s document management plan expects each business unit or program to designate a person or persons to lead the 
group’s documentation activities in DocuShare. These stewards are to be highly trained so they can perform DocuShare functions, provide guidance to others, 
enforce standards, and correct errors. The idea is to have fewer but better-trained users with full DocuShare permissions and functionality. While this will result in 
a less distributed workload, consistency and quality will improve, leading to greater tool utility. The Operations departments should follow a similar approach, and 
as much as possible, the workflow should occur within DocuShare.

Retaining the current Rubin DocuShare instance and enabling the Archive Server add-on requires both new and continuing costs and labor. The calendar year 
2020 renewal cost for Rubin’s current DocuShare instance was \$22,000, and Rubin's IT resources are expended to maintain the service, which is hosted on a 
physical server in Tucson. The archive server add-on carries a one-time cost of \$7,500 plus an additional \$1,400 per year for support, according to a December 
2020 quote from Xerox. Additionally, IT resources will be needed to configure the add-on, and some training likely will be required. Despite these considerations, 
the working group believes the benefits outweigh the costs, and the required resources are less burdensome than those that would be needed to adopt another 
repository.

A more detailed analysis of pro and cons is available in DocuShare Options Trade Study for the Documentation Working Group \href{https://docushare.lsst.org/docushare/dsweb/Get/Document-36788/DocuShareOptionsTradeStudy.pdf}{(Document-36788)}.
