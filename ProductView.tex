\subsection{Product View}

The \emph{Product View} is based on a common approach used in product management, a \emph{product tree diagram}.
The Documentation Working Group developed a framework customized for Rubin Observatory operations to facilitate the development of documentation and effective communication consistently project-wide.
The two primary purposes are to convey ownership and responsibility of technical documentation and more readily describe the complicated systems with context and categorization.
The Product View allows one to identify information as a normative source (e.g., one source of truth), then provide a manner which one can associate, link and cross reference documentation and informational dependencies from respective normative sources or references thereafter.
The expected primary users are managers, product owners, engineers, specialists, technicians, and users which want to search downwards in a system's hierarchy.

The Product View will consist of product trees developed by the owning \emph{Rubin Observatory Departments}, their technical teams and product owners.
Each department's product tree(s) should be constructed to best organize and compartmentalize the set of \emph{products} that make up the complicated systems under their purview.
These "products" can span all manner of objects; such as department-specific categories, documents, hardware (systems, servers, networking, etc.), data (data sets, validation tests, verification artifacts, etc.), software interface information (alarms and notifications between hardware or software systems, support data storage and metadata schema, etc.), or subject-matter expert support.
Each department or group is responsible to manage the product trees in addition to updating their technical documentation in a fashion that supports stakeholders' access to the associated information, interfaces and requirements.
Stakeholders (i.e., department technical groups, other departments, or external communities) should be able to retrieve current information with consistency and reliability.
In light of this need, the Documentation Working Group recommends all Product View trees be available via one of the project repositories --- it is suggested to create an appropriate lsst.io website so departments and groups changes can easily make changes via project-wide and the department's processes.

Products in the Product View are defined by the owning department in terms of a system or set of systems which can be grouped or decomposed into subsystems or individual components that is appropriate for them and their stakeholders.
The root-nodes and first set of leaf-nodes of each product tree are a predefined set of topics and categories created by the Documentation Working Group --- the \emph{generic categories}.
The generic categories were developed to capture critical operational aspects of a product while being extendable to subsystems of a product, where the subsystem may be a product with subsystems in its own right; i.e., a \emph{Level-1 product} can be comprised of multiple Level-2 or lower-level subsystems, and those subsystems which are products are \emph{Level-2+ products}.
Importantly, these terms do not correspond to the construction phase definition of Work Breakdown Structure (WBS) or colloquial "subsystem" used across the construction project.
It is at the departments discretion on how to best organize and characterize these products and their product trees to manage their systems and flow of information.
It is expected that all generic categories are applicable to the product trees associated with Level-1 and Level-2+ products alike.
While there will be exceptions for low-level systems, the Product View and its generic categories are designed to robustly capture construction completeness and operations readiness.
They should generate sufficient discussion between technical groups and the stakeholders to ensure key information is identified and provided.

\subsubsection{Departments for Rubin Observatory operations}

The Rubin Observatory departments are described as to their associated scope, systems and/or products descriptions, and contacts to identify managers, product owners, or other key personnel (e.g., organizational chart, contact list).
This section lists each department and respective scope.
If the scope description below does not already include the systems and products owned by the department, a high-level product tree diagram should be created.
Contact information should be available to individuals that require it, and it should be clear when it's appropriate to contact the individual or group.

\textbf{Director's Office} (DO) ---
The Vera C. Rubin Observatory Director's Office is responsible for the overall management of the observatory and the LSST survey, as well as fulfilling the mission of the observatory and realizing its vision.
The Director's Office includes a Directorate, Administrative Operations, Safety, Communications, and In-Kind Contributions teams.

\textbf{Observatory Operations} (OO) ---
The Chilean-based Rubin Observatory Operations Department is responsible for operating and maintaining the telescope, camera systems, and summit facilities in order to collect the raw imaging and housekeeping data needed by the LSST.
The primary tasks include maintaining the operating facilities, conducting the night-time survey operations, real-time assessment of image quality and observing efficiency, performing the daily calibration, and collecting and analyzing engineering data.

\textbf{Data Production} (DP) ---
The role of the Rubin Observatory Data Production department is to accept data from the Observatory's telescopes and ancillary systems; to process that data to generate science ready data products; to archive both raw data and derived data products; and, subject to approval from the Science Performance department and the Data Release Board, to make that data available to the scientific community.
The Data Production department will develop, maintain and operate the networks, compute and storage hardware, and software that constitutes the Rubin Observatory Data Management System for the duration of the operational period.

\textbf{System Performance} (SP) ---
Rubin Observatory System Performance department is responsible for ensuring that the LSST as a whole is proceeding with the efficiency and fidelity needed to achieve its science requirements at the end of the 10-year survey.
This includes the Wide-Fast-Deep (WFD) survey and all Special Programs (deep drilling fields and mini-surveys).
To meet this goal, the System Performance department will track and optimize the integrated performance of the entire system.
This includes the performance of the observatory and the progress of the survey with respect to its science objectives, the ability of the community to access and analyze the data and publish results on the four LSST science pillars at an appropriate rate, the evaluation of strategies for improving the survey strategy, and the development of mitigation strategies together with other relevant departments to minimize the impact of changes in the system performance on the overall LSST science.

\textbf{Education and Public Outreach} (EPO or EP) ---
The mission of the Rubin Observatory Education and Public Outreach program is to offer accessible and engaging online experiences that provide non-specialists access to, and context for, Rubin Observatory data so anyone can explore the Universe and be part of the discovery process.
EPO serves as a website that highlights and contextualizes the scientific power of Rubin Observatory for non-specialists and hosts all online resources.

\subsubsection{Generic Categories for the Product View}

The set of generic categories are provided as a basis to define and associate critical systems and objective elements for each department and their systems; they are not products in themselves.
They are designed by the Documentation Working Group to relay information required to support and evaluate operations readiness and operations specifically for Rubin Observatory in a concise manner by standardizing the distribution of information and products.
Further opportunities arise with a well designed set of product trees: clearly establish relationships and dependencies between systems, serve to introduce the department/system in a digestible manner, and create a more adaptable structure to organize and target information between technical groups internal or external to the department.

The generic categories are separated by five high-level categories --- Technical Design, Procedures, Safety and Emergency Response, Evaluation and Archival Documents.
Each high-level category includes a few subcategories, some of which have further subcategories.
The generic categories are meant to apply to a variety of systems (e.g., hardware-centric, software-centric, hardware and software distributed, process and protocol dedicated) such that information is associated with all categories and subcategories for practically all systems or subsystems described via a Product View tree.
It may be difficult at first for technical groups to perform a logical and relevant decomposition of the systems such that the generic categories are applicable to all product trees, especially if a system can change in different scenarios, contexts or states (e.g., one or more product trees could account for maintenance or on-sky operations of the Simonyi Survey Telescope, each state's different components and changing interdepartmental interfaces).
However, even in the case where the owner and stakeholders agree a category doesn't apply, the generic categories are a way to discuss, bound and described inter- and intra-departmental interfaces, requirements and key expectations.

It is important to note that the Product View is specifically intended to not have teams recreate or reproduce documentation or technical information.
The Product View should be used to collect and categorize information and documentation types, identify normative sources, and help identify and resolve any gaps between the group(s) and stakeholder(s).
Teams can and should refer to appropriate documentation or information therein to create product trees; this can sometimes take the form of a diagram or a narrative, and the representation may differ between the categories within a product tree.
Within the Product View as a whole, consisting of many product trees, it is intended systems and subsystems refer to higher- or lower-level product trees to reduce replication and the risk of confusing users accessing information.
Furthermore, the referential nature can be applied for interdepartmental information and dependencies such that it's clear which department owns the information and which departments utilize it, e.g., interface control documents (ICDs) and the N-squared diagram could be sufficient references.

Here are the generic categories:

\begin{small}

\begin{itemize}
  \item Technical Design
	\begin{itemize}

	  \item System Description
		\begin{itemize}
		  \item Description of System(s)
		  \item Definition of Sub-systems
		\end{itemize}

	  \item Technical Design Specifics
		\begin{itemize}
		  \item System-level and Intra-department System Interfaces
		  \item Sub-system Level Information
		  \item Inter-department Interfaces
		\end{itemize}
	  \item Access Interfaces (Physical and/or Software)
	\end{itemize}


  \item Procedures
	\begin{itemize}

	  \item Operational Procedures

	  \item Maintenance Procures
		\begin{itemize}
		  \item Preventive Maintenance
		  \item Reactive Maintenance
		  \item Turn-over Protocols (e.g., shift change, operational-to-maintenance status change)
		  \item Sub-system Isolation
		\end{itemize}

	  \item Software Access and Use Documentation for Users

	  \item Software Development Documentation for Developers

	  \item Manuals and Data Sheets
	\end{itemize}


  \item Safety and Emergency Response (System-level and relevant Sub-systems)
	\begin{itemize}

	  \item Emergency Procedures and Contacts

	  \item Hazards and Hazard Analysis

	  \item Mitigations and Verification Artifacts
		\begin{itemize}
		  \item Protocols
		  \item Energy Isolation
		\end{itemize}
	\end{itemize}


  \item Evaluation
	\begin{itemize}
	  \item Performance

	  \item Failure Effects and Failure Mode and Effects Analysis (FMEA)

	  \item Validation/Verification Test Plans

	  \item Test Reports
		\begin{itemize}
		  \item Technical Reports (Test and/or Analysis)
		  \item Verification Reports
		\end{itemize}
	\end{itemize}

  \item Archival Documents
	\begin{itemize}
	  \item Communications (e.g., Request for Information [RFI])
	  \item Construction Information, Unrelated to Commissioning/Operations
	\end{itemize}
\end{itemize}

\end{small}

\subsubsection{Examples of Product View trees}

Here are two examples of product tree referencing, using Auxiliary Telescope (AuxTel), LSSTCam and Commissioning Camera (ComCam).
Note that it would be beneficial to design product trees which take advantage of the similarities between LSSTCam and ComCam, even though there will be major differences, too.
For these three systems, the Access Interfaces category would include a common set of software, e.g., LSST Observing Visualization Environment (LOVE), Nublado (Jupyter) interface, Script Queue, Watcher.
This common set of software could be included in a higher-level product tree (potentially a Level-2 product under the Observatory Operations department) or as a referential Level-2+ product trees; then, not only is the information referenced to all three efficiently, but the software product tree(s) can be designed to easily indicate differences between the three systems, say, for the Software Development categories.
As a second example, the three systems interface with a set of external systems simultaneously connected to them, e.g., Engineering Facility Database (EFD), Environmental Awareness System (EAS), Global Interlock System (GIS).
These external systems could also be set up as referential product trees which are referenced by AuxTel, LSSTCam and ComCam, as well as others.

Figure \ref{fig:product-view-departments} depicts the Rubin Observatory departments and a subset of their systems and products into Level-1 products, with some Level-2+ products included.

\begin{figure}[t]
\caption{Example of Department Decomposition at System and Subsystem Level.}
\centering
\includegraphics[width=\textwidth]{product-view-departments-decomposition}
\label{fig:product-view-departments}
\end{figure}

Figure \ref{fig:product-view-auxtel-technical-design} is a more comprehensive example depicting the Technical Design high-level category from the generic categories.
Each subcategory (yellow) and each sub-subcategory (blue) is addressed such that the reader has an idea of what the system is, how it is used, and documentation with references therein for retrieving critical information.

\begin{figure}[pt]
\caption{Example of Technical Design Generic Category from Product View, for an Auxiliary Telescope Subsystem.}
\centering
\includegraphics[width=\textwidth]{product-view-auxtel-product-tree-technical-design}
\label{fig:product-view-auxtel-technical-design}
\end{figure}

Note that while it may seem Figure \ref{fig:product-view-departments} and Figure \ref{fig:product-view-auxtel-technical-design} are part of one example, that is not the intended purpose.
As described in the first example regarding AuxTel, LSSTCam and ComCam, it would be beneficial to organize the Product View to leverage the replicated information such as common software.
This was not considered for the two figures.


\begin{table}[]
\begin{tabular}{|l|l|l|l|}
\hline
 & \textbf{\begin{tabular}[c]{@{}l@{}}Rubin Observatory\\ Operations\end{tabular}} & \textbf{\begin{tabular}[c]{@{}l@{}}System\\ Performance\end{tabular}} & \textbf{\begin{tabular}[c]{@{}l@{}}Data\\ Production\end{tabular}} \\ \hline
\textbf{\begin{tabular}[c]{@{}l@{}}Technical\\ Design\end{tabular}} & \begin{tabular}[c]{@{}l@{}}ICDs/IDDs\\ Requirements\end{tabular} & \begin{tabular}[c]{@{}l@{}}Validation of\\ Science Platform\end{tabular} & \begin{tabular}[c]{@{}l@{}}Science Platform\\ Design and Operations\end{tabular} \\ \hline
\textbf{Procedures} & \begin{tabular}[c]{@{}l@{}}Day/Night Shift Turn-over,\\ Energy Isolation Protocols,\\ Calibration Procedures\end{tabular} & \begin{tabular}[c]{@{}l@{}}Predictive Analysis,\\ Hazard Validation\end{tabular} & \begin{tabular}[c]{@{}l@{}}Data Release and\\ Prompt Alerts\\ Processing\end{tabular} \\ \hline
\textbf{Evaluation} & \begin{tabular}[c]{@{}l@{}}Verification Test Plans,\\ Failure Effects,\\ Test Reports\end{tabular} & \begin{tabular}[c]{@{}l@{}}Failure Mode and\\ Effect Analysis\end{tabular} & \begin{tabular}[c]{@{}l@{}}Assertion\\ Testing\end{tabular} \\ \hline
\end{tabular}
\end{table}

